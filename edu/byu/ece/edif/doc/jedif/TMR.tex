
\section{JEdifTMR}
JEdif TMR takes the data from the analysis steps and creates a 
triplicated cell

\begin{verbatim}
> java edu.byu.ece.edif.jedif.JEdifTMR -h

Options:
  [-h|--help]
  [-v|--version]

  <input_file>
  [--ptmr <ptmr_file>]

  [(-o|--output) <output_file>]
  [--edif]

  [--tmr_suffix suffix1,suffix2,...,suffixN ]
  [--tmr_cell_name <tmr_cell_name>]
  [--tmr_domain_report <tmr_domain_report>]

  [--tmr_ports_file <tmr_ports_file>]

  [--log <logfile>]
  [--debug[:<debug_log>]]
  [(-V|--verbose) <{1|2|3|4|5}>]
  [--append_log]


\end{verbatim}

%%%%%%%%%%%%%%%%%%%%%%%%%%%%%%%%%%%%%%%%%%%%%%
\subsection{File Options}

\subsubsection{\texttt{<input\_file>}}
Filename and path to the jedif source file to be
Triplicated. This is the only required parameter.

\subsubsection{\texttt{--ptmr <ptmr\_file>}}
Name of the tmr data file. 

Default: \texttt{<inputfile>.ptmr} in the current working directory.

\subsubsection{\texttt{(-o|--output) <output\_file>}}
Filename and path to the output file. See the \texttt{--edif}
option

Default: \texttt{<inputfile>.<type>} in the current working directory.
(Extension depends on output type. Can be overridden by including
an extension on the command line)

\subsubsection{\texttt{--edif}}
Specifies whether to output an edf file or a jedif file.

Default: generate a jedif file.

%%%%%%%%%%%%%%%%%%%%%%%%%%%%%%%%%%%%%%%%%%%55
\subsection{TMR Options}
\subsubsection{\texttt{--tmr\_suffix}}
This option allows the user to specify the suffixes that the tool adds to each
of the replicated design elements (nets, cell instances, etc.). The user must
specify all three suffixes (in a comma-separated list) to be used for each of
the three TMR domains. In addition, each of the domains must be unique to avoid
naming conflicts. One of suffixes may be left blank to signify that no suffix
should be added to the elements of that domain.

As an example, if the user specified the following list of suffixes:

\texttt{--tmrSuffix \_0,\_1,\_2}

a net in the original design called \texttt{reset} would be triplicated as:

\texttt{reset\_0, reset\_1, and reset\_2}.

The default suffixes (if the \texttt{--tmrSuffix} option is not used) are 
\texttt{\_TMR\_0, \_TMR\_1, and \_TMR\_2}.

\subsubsection{\texttt{--tmr\_cell\_name}}
 Specifies the name of the TMR'd cell


\subsubsection{\texttt{--domain\_report <domainReport>}}
The name of the domain report file. The domain report lists the domain (0, 1, 
or 2) of each cell instance in the resulting EDIF file. This report is used as 
input to the LANL RadDRC half-latch removal tool. 

Default: \texttt{BL-TMR\_domain\_report.txt}.

\subsubsection{\texttt{--tmr\_ports\_file <tmr\_ports\_file>}}
The name of the tmr\_ports file. It consists of a list of ports to
triplicate followed by three names for the new ports, in this format:

\texttt{clk:clk0,clk1,clk2}
\texttt{rst0:rst0,rst1,rst2}

Default: none


\subsection{Logging options}

\subsubsection{\texttt{--log <logfile>}}
Specifies a file for logging output (default: sterilize.log)

\subsubsection{\texttt{--debug[:<debug\_log>]}}
Specifies a file for logging the debuggin output.If no file
specified, debug output is printed to the log file.

\subsubsection{\texttt{(-V|--verbose) <$\{1|2|3|4|5$\}>}}
Sets the verbosity level:
1 prints only errors, 
2 warnings, 
3 normal, 
4 log to stdout. 
5 prints debug information. 
(default: 3)

\subsubsection{\texttt{--append\_log }}
Append to the logfile instead of replacing it.


