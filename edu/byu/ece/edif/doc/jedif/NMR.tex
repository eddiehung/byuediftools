\section{JEdifNMR}
JEdifNMR performs the replication selected by previously run
tools. Information about what to replicate and where to insert
voters/detectors is obtained from the replication description file
(.rdesc) created by the previous steps.

\begin{verbatim}
> java edu.byu.ece.edif.jedif.JEdifNMR
Options:
  [-h|--help]
  [-v|--version]

  <input_file>
  (-r|--rep_desc) <rep_desc>
  [(-o|--output) <output_file>]
  [--edif]

  [(-p|--part) <part>]
  
  [--write_config <config_file>]
  [--use_config <config_file>]

  [--log <logfile>]
  [--debug[:<debug_log>]]
  [(-V|--verbose) <{1|2|3|4|5}>]
  [--append_log]
\end{verbatim}

%%%%%%%%%%%%%%%%%%%%%%%%%%%%%%%%%%%%%%%%%%%%%%
\subsection{File Options}

\subsubsection{\texttt{<input\_file>}}
Filename and path to the jedif source file to be
replicated.

\subsubsection{\texttt{(-r|--rep\_desc) <rep\_desc>}}
Filename and path to the replication description (.rdesc) file containing the
replication information.

\subsubsection{\texttt{(-o|--output) <output\_file>}}
Filename and path to the output file. If the given filename ends in .edf or if
the \texttt{--edif} option is specified, an EDIF file will be generated.
Otherwise, the replicated circuit will be output in .jedif format.

\subsubsection{\texttt{--edif}}
Specifies that an EDIF (.edf) file should be generated instead of a .jedif file.

%%%%%%%%%%%%%%%%%%%%%%%%%%%%%%%%%%%%%%%%%%%55


\subsection{Target Technology and Part Options}

\subsubsection{\texttt{--technology <techname>}}
Target architecture for the triplicated design. Used to take into account 
various technology-specific properties. This argument is \emph{not} 
case-sensitive.

Valid technologies: \texttt{Virtex} and \texttt{Virtex2}. Default: 
\texttt{Virtex}.

\subsubsection{\texttt{--part <partname>}}
Target architecture for the triplicated design. Used to take into account 
part-specific properties, including the number of resources available 
in each part. Valid parts include all parts from the \emph{Virtex} and 
\emph{Virtex2} product lines, represented as a concatenation of the part name 
and package type. For example, the ``Xilinx Virtex 1000 FG680'' is represented 
as \texttt{XCV1000FG680}. This argument is \emph{not} case-sensitive.

Default: \texttt{xcv1000fg680}.
% TODO: Add list of all supported part numbers. All parts listed in
% XilinxVirtexDeviceUtilizationTracker.java and
% XilinxIIVirtexDeviceUtilizationTracker.java

%%%%%%%%%%%%%%%%%%%%%%%%%%%%%%%%%%%%%%%%%%%%%%
\subsection{Configuration File Options}
\label{config options}
The BLTmr tools can use configuration files in place of command-line parameters. 
If a parameter is specified in a configuration file, it will be passed to the 
BLTmr tool, unless it is overridden by the same argument on the command-line. 

\subsubsection{\texttt{--useConfig <config\_file>}}
\label{useConfig}
Specify a configuration file from which to read parameters.

\subsubsection{\texttt{--writeConfig[:<config\_file>]}}
Write the current set of command-line parameters to a configuration file and 
exit. The parameters will be parsed to ensure they are valid, but the BLTmr tool
will not run.  Note that only the parameters on the command-line are stored in
the configuration file. When using \texttt{--writeConfig}, any use of
\texttt{--useConfig} is ignored. This is to prevent complicated cascades
of configuration files combined with command-line options.

Examples:
\begin{itemize}
  \item \texttt{--writeConfig:JonSmith.conf} will write the command-line 
  parameters to the file \texttt{JonSmith.conf} in the current directory.
  \item \texttt{--writeConfig:/usr/lib/BLTmr/common.conf} will write the
  command-line parameters to the file \texttt{/usr/share/BLTmr/common.conf}. 
  \item See section \ref{using config}, ``Using Configuration Files,'' for
  more information.
\end{itemize}

\subsection{Logging options}

\subsubsection{\texttt{--log <logfile>}}
Specifies an alternate file for logging output.

\subsubsection{\texttt{--debug[:<debug\_log>]}}
Specifies a file for logging the debugging output.If no file
specified, debug output is printed to the log file.

\subsubsection{\texttt{(-V|--verbose) <$\{1|2|3|4|5$\}>}}
Sets the verbosity level:
1 prints only errors, 
2 warnings, 
3 normal, 
4 log to stdout. 
5 prints debug information. 
(default: 3)

\subsubsection{\texttt{--append\_log }}
Append to the logfile instead of replacing it.

