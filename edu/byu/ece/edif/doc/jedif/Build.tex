\section{JEdifBuild Options}
Options can be specified on the command line or in a configuration file in any 
order. This section describes each of these options in detail, which are 
summarized below:\footnote{This list is obtainable with the \texttt{--help} 
option. Here, newlines have been inserted to separate the options into 
categories.}

\begin{verbatim}
> java byucc.edif.tools.merge.JEdifBuild --help
Options:
  [-h|--help]
  [-v|--version]

  <input_file>
  [(-o|--output) <output_file>]

  [(-d|--dir) dir1,dir2,...,dirN ]
  [(-f|--include_file) include_file1,include_file2,...,include_fileN ]

  [--no_flatten]
  [--no_delete_cells]
  [--replace_srls]
  [--remove_rlocs]

\end{verbatim}

%%%%%%%%%%%%%%%%%%%%%%%%%%%%%%%%%%%%%%%%%%%%%%
\subsection{File options: input, output, etc.}
The following options specify the top-level input EDIF file, any auxiliary EDIF
files, and the destination EDIF file.

\subsubsection{\texttt{<input\_file>}}
Filename and path to the EDIF source file containing the top-level cell to be 
converted. This is the only required parameter.

Allowed filename extensions are:
\begin{itemize}
  \item Parsable Edif: edn,edf,ndf
  \item Binary Edif (Blackboxes): ngc,ngo
  \item Blackbox Utilization: bb
\end{itemize}

Parsable EDIF files will be parsed and included in the algorithms.
Binary EDIF files are not parsable, but the program recognizes them as
blackboxes, and will not complain about not finding the entity.
Blackbox utilization files allow the user to specify the resource use
of the blackboxes to help in the utilization estimate and partial tmr
algorithms.  The file format is ``Resource:Number''.  Below is an
example:

myblackbox.bb:\\
\begin{verbatim}
  BRAM:1
  FF:400
  LUT:100
\end{verbatim}

This entity, named myblackbox uses 1 BRAM, 400 Flipflops and 100 LUTS

\subsubsection{\texttt{(-o|--output) <output\_file>}}
Filename and path to the jedif output file.

Default: \texttt{BLTmr.edf} in the current working directory.

\subsubsection{\texttt{(-d|--dir) dir1,dir2,\ldots,dir3}}
Comma-separated list of directories containing external EDIF files referenced 
by the top-level EDIF file. The current working directory is included by 
default and need not be specified. There can be multiple \texttt{-d} options.

Example: \texttt{-d aux\_files,/usr/share/edif/common -d
  moreEdifFiles/}

\subsubsection{\texttt{(-f|--file) file1,file2,\ldots,fileN}}
Similar to the previous option, but rather than specifying directories to 
search, each external EDIF file is named explicitly---including the path to the 
file. There can be multiple \texttt{-f} options. 

Example: \texttt{-f multBox.edn,src/adder.edf -f /usr/share/edif/blackBox.edf}.

\subsection{Maintenance Commands}
The following options allow some control over that happens during the 
conversion process

\subsubsection{\texttt{--no\_flatten}}
By default JEdifBuild will flatten the EDIF files. Flattening is required by the 
TMR tools, but other applications my wish to have retail the hierarchical design.

\subsubsection{\texttt{--no\_delete\_cells}}
By default JEdifBuild will remove unused cells, to reduce the size of the final
.jedif file. However, the user can request that these cells be retained for 
future use.

\subsubsection{\texttt{--replace\_srls}}
By default JEdifBuild will not remove LUT RAMs acting as SRLs in the design.  These
LUT RAMs can cause problems as they are not scrubbable.  This option tells the 
tool to replace these LUT RAMs with actual SRLs.

\subsubsection{\texttt{--remove\_rlocs}}
Setting this option will tell JEdifBuild to remove ALL RLOCS in the design.  Some
RLOCS can be problems for different versions of the hardware.

%
% Words to be ignored by the spell-checker:
%

% LocalWords:  BYU LANL BLTmr EDIF FPGA TMR OBUF IBUF BUFG IBUFG LUTs
% LocalWords:  SCC SCCs FFs UCF Xilinx java JHDL netlister IOB IBUFs
% LocalWords:  OBUFs logfile INOUT TMR'd tmr txt JEdifBuild jedif edn edf dir


