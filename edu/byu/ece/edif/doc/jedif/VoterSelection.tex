\section{JEdifVoterSelection}
JEdifVoterSelection determines the locations where voters will be
inserted into a triplicated design (or triplicated portions of a
design). Voter locations are determined using a feedback cutset
algorithm and rules for voting where downscaling is necessary. The
results are added into the replication description file (.rdesc).

At times, the user may wish to force voter insertion on certain nets
and disable voter insertion on others. This can be accomplished by
inserting \texttt{`force\_restore'} and \texttt{`do\_not\_restore'}
properties on selected nets in the .edf file as follows:\\
\texttt{(property force\_restore (boolean (true)))}\\
\texttt{(property do\_not\_restore (boolean (true)))}\\
\begin{verbatim}
>java edu.byu.ece.edif.jedif.JEdifVoterSelection
Options:
  [-h|--help]
  [-v|--version]
  
  <input_file>
  (-r|--rep_desc) <rep_desc>
  (-c|--c_desc) <c_desc>
    
  [--after_ff_cutset]
  [--before_ff_cutset]
  [--connectivity_cutset]
  [--basic_decomposition]
  [--highest_fanout_cutset]
  [--highest_ff_fanout_cutset]
  [--highest_ff_fanin_input_cutset]
  [--highest_ff_fanin_output_cutset]
  
  [--write_config <config_file>]
  [--use_config <config_file>]

  [--log <logfile>]
  [--debug[:<debug_log>]]
  [(-V|--verbose) <{1|2|3|4|5}>]
  [--append_log]
\end{verbatim}
%%%%%%%%%%%%%%%%%%%%%%%%%%%%%%%%%%%%%%%%%%%%%%%
\subsection{File Options}

\subsubsection{\texttt{<input\_file>}}
Filename and path to the .jedif source file.

\subsubsection{\texttt{(-r|--rep\_desc) <rep\_desc>}}
Filename and path to the replication description (.rdesc) file to be modified.

\subsubsection{\texttt{(-c|--c\_desc) <c\_desc>}}
Filename and path to the circuit description (.cdesc) file generated by
JEdifAnalyze.

%%%%%%%%%%%%%%%%%%%%%%%%%%%%%%%%%%%%%%%%%%%%%%%%%%
\subsection{Cutset Algorithms}

Synchronization voters are essential in FPGA circuits that use TMR
because they ensure that the internal state of all three TMR
replicates are synchronized after configuration scrubbing.  Adding
synchronization voters in a design manually, however, is a difficult
and error prone process.  This tool uses automated cutset algorithms
for selecting synchronization voter locations and inserting them in
the design.

Synchronization voter insertion algorithms must determine a set of
nets within a design that cuts {\em all} feedback in the
design. Voters are placed on each of these nets to ensure
synchronization voting occurs within the feedback structures of a
design.  Determining a set of voter locations that satisfy this
constraint is an instance of the feedback edge set (FES) problem. The
algorithms used in this tool solve the FES problem for voter insertion
in a way that avoids illegal cut locations. In addition, many of the
algorithms employ heuristics based on FPGA architecture that attempt
to minimize circuit area or timing impact.

The first two algorithms are very simple voter insertion algorithms
that solve the problem in a local manner. These will be followed by
five algorithms based on strongly connected component (SCC)
decomposition that attempt to meet the constraints while using fewer
voters and applying timing-based heuristics. The run-time complexity
of each algorithm will be given in terms of $|V|$ (the number of nodes
in the circuit graph) and $|E|$ (the number of edges in the circuit
graph).

\subsubsection{\texttt{--before\_ff\_cutset}}
The \emph{Voters Before Every Flip-Flop} algorithm places a voter
before the data input of every flip-flop in a circuit.  While this
algorithm does not detect feedback in the circuit, this approach will
intersect sequential feedback as all synchronous feedback contains at
least one flip-flop.  This algorithm requires only a simple analysis
of the circuit and runs in $O(|V|)$ time. Although it is simple, the
algorithm ensures that only one set of triplicated voters can be
placed in a single timing path, which helps reduce the negative timing
impact of voter insertion. A simple timing path is the path from one
flip-flop to another; when voters are placed only directly before each
flip-flop, it is impossible to have more than one flip-flop in a
single timing path. The disadvantage of this algorithm is that it adds
far more voters than necessary.

\subsubsection{\texttt{--after\_ff\_cutset}}
The \emph{Voters After Every Flip-Flop} algorithm places a voter after
the output of each flip-flop in a circuit. As with the \emph{Voters
Before Every Flip-Flop} algorithm, this algorithm is guaranteed to
intersect every cycle with a voter because in standard synchronous
circuits, each cycle must have at least one flip-flop. This algorithm
also runs in $O(|V|)$ time. Like the previous algorithm, it ensures
that only a single set of triplicated voters is inserted in each
timing path. In a study with 15 benchmark designs, this algorithm
provided the best average timing result.

\subsubsection{\texttt{--connectivity\_cutset}}
This is the original algorithm that removes arbitray feedback edges
until all feedback is cut. This option has been shown to produce
inferior results in general to the others but in some few cases it
\emph{may} give better timing results (this is not likely in
real-world designs).

\subsubsection{\texttt{--basic\_decomposition}}
The \emph{Basic SCC Decomposition Algorithm} uses temporary
information obtained during SCC decomposition to completely cut each
SCC in a single step.  This approach computes SCC decomposition using
Kosaraju's algorithm which uses two depth-first searches (DFS). The
DFS {\em back edges} computed during Kosaraju's algorithm are used to
remove {\em all} cycles in the SCC (removing all DFS back edges from
an SCC removes all feedback)\footnote{In some cases, this approach
selects edges that correspond to illegal cut locations.  When this
happens, special case exception logic is used to find a set of edges
that completely cuts the feedback in the SCC.}. This algorithm runs
quickly on average, but typically induces poor timing performance in
the resulting circuit. The algorithm's runtime complexity is $O(|V|^2
+ |V||E|)$.

\subsubsection{\texttt{--highest\_fanout\_cutset}}
The \emph{Highest Fanout SCC Decomposition Algorithm} is an attempt to
reduce the number of voters used to intersect the cycles of a
circuit. It uses a heuristic that suggests that a significant amount
of feedback can be cut by inserting voters on a single net with high
fanout. At each iteration, the SCC in question is analyzed to find the
node with the highest legal cut fanout. The legal cut output edges
from this node are then removed from the graph. In this manner, edge
removal is prioritized with high fanout nets. The result is that this
algorithm cuts all of a circuit's feedback using fewer voters than
most other algorithms. The algorithm runs in $O(|V|^2|E|)$ time.

\subsubsection{\texttt{--highest\_ff\_fanout\_cutset}}
The \emph{Highest Flip-Flop Fanout SCC Decomposition Algorithm} is
similar to the previous algorithm but identifies high fanout nets that
originate from flip-flops. This algorithm has two priorities:
inserting a small number of voters and reducing the negative impacts
of voter insertion on timing performance. When more than one set of
voters is inserted in a single timing path (i.e. a path from one
register to the next), the voters negatively affect timing performance
more than is necessary.  For each SCC processed by this algorithm, the
flip-flop with the highest legal cut fanout in the SCC is
determined. The legal cut output edges from this node are
removed. Since a timing path consists of the logic from one flip-flop
to the next, inserting voters only directly after flip-flop outputs
ensures that at most one voter will be inserted per timing path. The
runtime of this algorithm is the same as the previous algorithm,
$O(|V|^2|E|)$. In a study with 15 benchmark designs, this algorithm
provided the second best average timing result, and the best
combination overall of area and timing results.

\subsubsection{\texttt{--highest\_ff\_fanin\_input\_cutset}}
Inserting triplicated voters just before a flip-flop with high fan-in
can cut a significant amount of feedback.  In this algorithm,
flip-flop fan-in is defined as the number of nets that directly or
indirectly feed into the data input of a flip-flop going up to five
levels backwards as computed by a depth-limited DFS traversal.  For
each SCC being processed, this algorithm finds the flip-flop in the
SCC with the highest fan-in that also has a data input edge that is a
legal voter location and removes its data input edge. The run time of
this algorithm is $O(|V|^3 + 2|V|^2|E| + |V||E|^2)$. The intent is the
same as that of the previous algorithm: to reduce the number of voters
used to cut feedback and limit voters to one set of triplicated voters
per timing path.

\subsubsection{\texttt{--highest\_ff\_fanin\_output\_cutset}}
This algorithm is very similar to the preceding algorithm and has the
same objectives. It is different only in that it inserts voters
directly after flip-flops with high fan-in instead of directly
before. For each SCC being processed, the algorithm finds the
flip-flop in the SCC with the highest fan-in and a legal voter
location output edge and removes the data output edge. The runtime is
the same as that of the previous algorithm, $O(|V|^3 + 2|V|^2|E| +
|V||E|^2)$. In a study with 15 benchmark designs, this algorithm
provided the third best average timing result.

%%%%%%%%%%%%%%%%%%%%%%%%%%%%%%%%%%%%%%%%%%%%%%%%%%

%%%%%%%%%%%%%%%%%%%%%%%%%%%%%%%%%%%%%%%%%%%%%%
\subsection{Configuration File Options}
\label{config options}
The BLTmr tools can use configuration files in place of command-line parameters. 
If a parameter is specified in a configuration file, it will be passed to the 
BLTmr tool, unless it is overridden by the same argument on the command-line. 

\subsubsection{\texttt{--useConfig <config\_file>}}
\label{useConfig}
Specify a configuration file from which to read parameters.

\subsubsection{\texttt{--writeConfig[:<config\_file>]}}
Write the current set of command-line parameters to a configuration file and 
exit. The parameters will be parsed to ensure they are valid, but the BLTmr tool
will not run.  Note that only the parameters on the command-line are stored in
the configuration file. When using \texttt{--writeConfig}, any use of
\texttt{--useConfig} is ignored. This is to prevent complicated cascades
of configuration files combined with command-line options.

Examples:
\begin{itemize}
  \item \texttt{--writeConfig:JonSmith.conf} will write the command-line 
  parameters to the file \texttt{JonSmith.conf} in the current directory.
  \item \texttt{--writeConfig:/usr/lib/BLTmr/common.conf} will write the
  command-line parameters to the file \texttt{/usr/share/BLTmr/common.conf}. 
  \item See section \ref{using config}, ``Using Configuration Files,'' for
  more information.
\end{itemize}

\subsection{Logging options}

\subsubsection{\texttt{--log <logfile>}}
Specifies a file for logging output (default: sterilize.log)

\subsubsection{\texttt{--debug[:<debug\_log>]}}
Specifies a file for logging the debuggin output.If no file
specified, debug output is printed to the log file.

\subsubsection{\texttt{(-V|--verbose) <$\{1|2|3|4|5$\}>}}
Sets the verbosity level:
1 prints only errors, 
2 warnings, 
3 normal, 
4 log to stdout. 
5 prints debug information. 
(default: 3)

\subsubsection{\texttt{--append\_log }}
Append to the logfile instead of replacing it.

