\section{JEdifVoterSelection}
JEdifVoterSelection determines the locations where voters will be
inserted into a triplicated design (or triplicated portions of a
design). Voter locations are determined using a feedback cutset
algorithm and rules for voting where downscaling is necessary. The
results are added into the replication description file (.rdesc).

At times, the user may wish to force voter insertion on certain nets and
disable voter insertion on others. This can be accomplished by inserting
\texttt{`force\_restore'} and \texttt{`do\_not\_restore'} properties on
selected nets in the .edf file as follows:\\
\texttt{(property force\_restore (boolean (true)))}\\
\texttt{(property do\_not\_restore (boolean (true)))}\\
\begin{verbatim}
>java edu.byu.ece.edif.jedif.JEdifVoterSelection
Options:
  [-h|--help]
  [-v|--version]
  
  <input_file>
  (-r|--rep_desc) <rep_desc>
  (-c|--c_desc) <c_desc>

  [--highest_ff_fanout_cutset]
  [--highest_fanout_cutset]
  [--connectivity_cutset]

  [--write_config <config_file>]
  [--use_config <config_file>]

  [--log <logfile>]
  [--debug[:<debug_log>]]
  [(-V|--verbose) <{1|2|3|4|5}>]
  [--append_log]
\end{verbatim}
%%%%%%%%%%%%%%%%%%%%%%%%%%%%%%%%%%%%%%%%%%%%%%%
\subsection{File Options}

\subsubsection{\texttt{<input\_file>}}
Filename and path to the .jedif source file.

\subsubsection{\texttt{(-r|--rep\_desc) <rep\_desc>}}
Filename and path to the replication description (.rdesc) file to be modified.

\subsubsection{\texttt{(-c|--c\_desc) <c\_desc>}}
Filename and path to the circuit description (.cdesc) file generated by
JEdifAnalyze.

%%%%%%%%%%%%%%%%%%%%%%%%%%%%%%%%%%%%%%%%%%%%%%%%%%
\subsection{Cutset Algorithms}
This tool can use several different algorithms to determine where to place
voters so that the voters cut all the feedback in the design. 

\subsubsection{\texttt{--highest\_ff\_fanout\_cutset}}
This algorighm finds the flip-flop with the highest fanout in each SCC and 
places a voter on the output. This algorithm has proven very good at reducing 
the number of paths that have more than one voter between flip-flops and gives
good timing and area results. This is the default algorithm.

\subsubsection{\texttt{--highest\_fanout\_cutset}}
This algorithm finds the instance with the highest fanout in each SCC.
It then places a voter on this output. This algorithm has proven worse 
at reducing the number of voters between flip-flops.

\subsubsection{\texttt{--connectivity\_cutset}}
This is the original algorithm that removes arbitray feedback edges until all
feedback is cut. This option has been shown to produce inferior results in
general to the other two but in some few cases it \emph{may} give better timing
results (not likely in real-world designs).

\subsubsection{\texttt{--basic\_decomposition}}
This is a basic SCC decomposition similar to the original connectivity cutset.
It removes feedback edges in an arbitrary order until all feedback is cut. It
varies slightly from the original connectivity cutset algorithm in the method
used to select edges to cut. This algorithm is experimental.

\subsubsection{\texttt{--highest\_ff\_fanin\_cutset}}
This is the highest flip-flop fanin cutset. It finds the flip-flop with the
highest fan-in in each SCC (where fan-in is defined as the number of nets
feeding into the data input up to 5 levels backwards in a reverse DFS) and
places a voter just before the input. This algorithm is experimental.

\subsubsection{\texttt{--after\_ff\_cutset}}
This cutset algorithm simply places a voter after each flip-flop in a circuit.
This algorithm is experimental.

\subsubsection{\texttt{--before\_ff\_cutset}}
This cutset algorithm simply places a voter before each flip-flop in a circuit.
This algorithm is experimental.

%%%%%%%%%%%%%%%%%%%%%%%%%%%%%%%%%%%%%%%%%%%%%%%%%%

%%%%%%%%%%%%%%%%%%%%%%%%%%%%%%%%%%%%%%%%%%%%%%
\subsection{Configuration File Options}
\label{config options}
The BLTmr tools can use configuration files in place of command-line parameters. 
If a parameter is specified in a configuration file, it will be passed to the 
BLTmr tool, unless it is overridden by the same argument on the command-line. 

\subsubsection{\texttt{--useConfig <config\_file>}}
\label{useConfig}
Specify a configuration file from which to read parameters.

\subsubsection{\texttt{--writeConfig[:<config\_file>]}}
Write the current set of command-line parameters to a configuration file and 
exit. The parameters will be parsed to ensure they are valid, but the BLTmr tool
will not run.  Note that only the parameters on the command-line are stored in
the configuration file. When using \texttt{--writeConfig}, any use of
\texttt{--useConfig} is ignored. This is to prevent complicated cascades
of configuration files combined with command-line options.

Examples:
\begin{itemize}
  \item \texttt{--writeConfig:JonSmith.conf} will write the command-line 
  parameters to the file \texttt{JonSmith.conf} in the current directory.
  \item \texttt{--writeConfig:/usr/lib/BLTmr/common.conf} will write the
  command-line parameters to the file \texttt{/usr/share/BLTmr/common.conf}. 
  \item See section \ref{using config}, ``Using Configuration Files,'' for
  more information.
\end{itemize}

\subsection{Logging options}

\subsubsection{\texttt{--log <logfile>}}
Specifies a file for logging output (default: sterilize.log)

\subsubsection{\texttt{--debug[:<debug\_log>]}}
Specifies a file for logging the debuggin output.If no file
specified, debug output is printed to the log file.

\subsubsection{\texttt{(-V|--verbose) <$\{1|2|3|4|5$\}>}}
Sets the verbosity level:
1 prints only errors, 
2 warnings, 
3 normal, 
4 log to stdout. 
5 prints debug information. 
(default: 3)

\subsubsection{\texttt{--append\_log }}
Append to the logfile instead of replacing it.

