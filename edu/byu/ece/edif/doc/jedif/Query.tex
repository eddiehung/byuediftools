\section{JEdifQuery}
JEdifQuery is used to query the contents of a .jedif file and to
provide summary information about the EDIF design contained within.
\begin{verbatim}
>java edu.byu.ece.edif.jedif.JEdifQuery
Options:
  [-h|--help]
  [-v|--version]

  <input_file>

  [--libraries]
  [--ports cell1,cell2,...,cellN ]
  [--instance_list cell_type1,cell_type2,...,cell_typeN ]
  [--cells_in_library library1,library2,...,libraryN ]
  [--subcells cell1,cell2,...,cellN ]
  [--cell_nets cell1,cell2,...,cellN ]
  [--primitive_list cell1,cell2,...,cellN ]
  [--dangling_nets]
  [--count_persistent_FFs]
\end{verbatim}

\subsection{File Options:}
\subsubsection{\texttt{<input\_file>}}
Filename and path to the .jedif source file containing the EDIF environment to
be queried. This is the only required parameter.

\subsection{Querying Options}
By default (if only an input filename is provided), JEdifQuery will
display the name of the top-level cell of the design, the number of
nets and instances contained in the design, and the top-level ports
of the design. Further information can be obtained by providing one
or more of the following command-line options:

\subsubsection{\texttt{--libraries}}
Displays a list of all libraries contained in the design.

\subsubsection{\texttt{--ports cell1,cell2,/ldots,cellN}}
All ports of specified cells contained in the design will be displayed.

\subsubsection{\texttt{--instance\_list
cell\_type1,cell\_type2,\ldots,cell\_typeN}}
The names of all instances of specified cells contained in the design will be displayed.

\subsubsection{\texttt{--cells\_in\_library library1,library2,\ldots,libraryN}}
Names of all cells contained in the specified libraries will be displayed.

\subsubsection{\texttt{--subcells cell1,cell2,\ldots,cellN}}
Names of all instances within the specified cells will be displayed.

\subsubsection{\texttt{--cell\_nets cell1,cell2,\ldots,cellN}}
Names of all nets within the specified cells will be displayed.

\subsubsection{\texttt{--primitive\_list cell1,cell2,\ldots,cellN}}
Names of all primitives within the cell will be displayed. Note that
this option only displays the names of primitives that are immediate
subcells of the specified cells; the name of a primitive that is within
a subcell of a specified subcell will not be displayed.

\subsubsection{\texttt{--dangling\_nets}}
This option displays a list of all dangling nets contained in the
design, along with the cell and library in which those nets are
contained.
