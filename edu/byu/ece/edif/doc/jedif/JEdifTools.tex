%%%%%%%%%%%%%%%%%%%%%%%%%%%%%%%%%%%%%%%%%%%%%%%%%%%%%%%%%%%%%%%%%%%%%%%%%%%%%%%
% $Id$
%
% Documentation file for the BYU-LANL Triple Modular Redundancy (BLTmr) Tool.
%
% Author: Brian Pratt <bhpratt@gmail.com>
%         James Carroll <jcarroll@byu.net>
%%%%%%%%%%%%%%%%%%%%%%%%%%%%%%%%%%%%%%%%%%%%%%%%%%%%%%%%%%%%%%%%%%%%%%%%%%%%%%%

\documentclass[english]{article}
%\usepackage[T1]{fontenc}
\usepackage{fullpage}
\usepackage{amsmath}
\usepackage{epsfig}
\usepackage{graphicx}
\usepackage{verbatim}
\usepackage{moreverb}
%\usepackage[latin1]{inputenc}
\IfFileExists{url.sty}{\usepackage{url}}
                      {\newcommand{\url}{\texttt}}
% use this instead of pdfgraphcompat
% Check for PDFLaTeX
\newif\ifpdf 
\ifx\pdfoutput\undefined 
   \pdffalse % we are not running PDFLaTeX 
\else 
   \pdfoutput=1 % we are running PDFLaTeX 
   \pdftrue 
\fi


% This processes the file using the 'hyperref.sty' package when
% PDFLaTeX is used.  This adds internal hyperlinks throughout the
% document in the generated PDF.
\ifpdf
   \usepackage[colorlinks={true},
     urlcolor=rltblue,       % \href{...}{...} external (URL)
     filecolor=rltgreen,     % \href{...} local file
     linkcolor=rltred,       % \ref{...} and \pageref{...}
     pdftitle={BYU-LANL Triple Modular Redundancy Usage Guide Version 0.3.3 -  15 Jan 2008},%
     pdfauthor={BYU Configurable Computing Lab},%
     pdfproducer={pdfLaTeX},%
     %pdfadjustspacing=1,
     pdftex]{hyperref}
\fi

% Define colors used by hyperref
\usepackage{color}
\definecolor{rltred}{rgb}{0.75,0,0}
\definecolor{rltgreen}{rgb}{0,0.5,0}
\definecolor{rltblue}{rgb}{0,0,0.75}


\vfuzz2pt % Don't report over-full v-boxes if over-edge is small
\hfuzz2pt % Don't report over-full h-boxes if over-edge is small

% 1: label, 2: path, 3: filename except extension (minus .png or .eps), 4: caption
\newcommand\figurecaption[4]{
\begin{figure}[ht]
  \centering
  \includegraphics[width=1.0\linewidth]{#2/#3}
  \parbox{1.0\linewidth}{\caption{\label{#1}#4}}
\end{figure}
}

\makeatletter

%%%%%%%%%%%%%%%%%%%%%%%%%%%%%% LyX specific LaTeX commands.
%% Bold symbol macro for standard LaTeX users
\providecommand{\boldsymbol}[1]{\mbox{\boldmath $#1$}}

\usepackage{babel}
\makeatother

\title{BYU-LANL Triple Modular Redundancy \\ Usage Guide \\ ~ \\
  Version 0.3.3 - 15 Jan 2008 \\  }
  
\author{Brigham Young University \\ Configurable Computing Lab}

\date{\today}

\begin{document}

\maketitle

\newpage
\tableofcontents
\newpage

%%%%%%%%%%%%%%%%%%%%%%%%%%%%%%%%%%%%%%%%%%%%%%%%%%%%%%%%%%%%
\section{Introduction}
The BYU-LANL Triple Modular Redundancy (BLTmr) Tool is an EDIF-based tool to 
insert redundancy in an FPGA design in order to increase reliability. Triple 
modular redundancy (TMR) is applied to the EDIF input file according to the 
options chosen by the user. Partial TMR focuses on ``persistent'' components of 
the design in order to get the ``most bang for your buck.''

\section{TMR Toolflow}
The Tool is split into several subtools. This allows the user to adjust various 
command-line options in one phase, and then move onto the next phase, without
having to wait for the entire TMR process to complete.

\subsection{JEdifBuild}
JEdifBuild creates merged netlists in a .jedif file format from 
multiple .edif files. The .jedif file format is used by all the TMR tools.

\subsection{JEdifNetList}
JEdifNetList will convert a .jedif file into an .edf file for use with other
standard edif tools 

\subsection{JEdifSterilize}
The first Tool in the TMR chain. This tool removes HalfLatches and FMAPs. If the 
design does not have any, this tool is optional.

\subsection{JEdifTMRAnalysis}
This tool determines what parts of the cell to triplicate, and where to put voters,
based on user input and the size of the device.

\subsection{JEdifCutset}
This tool finds all the feedback in the design, and creates of list of voters
to be instered to interupt all feedback.

\subsection{JEdifMoreFrequentVoting}
\emph{Optional:} Inserts extra voters for more frequent voting within
the design.

\subsection{JEdifTMR}
Takes information about voter placement from the previous steps, and
creates a new design incorperating that information. Can output either
a JEdif or an EDIF file.

\subsection{EdifTMR}
The EdifTMR tools combines all the tools in the JEdifTMR toolflow into
one easy step. If you already know exactly what setting you wish use
to perform TMR, this tool will be faster.

\subsection{JEdifQuery}
JEdifQuery is a tool used to query the contents of a jedif file and to 
provide summary information about the EDIF design contained within.

\subsection{Other JEdif tools}
\subsubsection{JEdifClockDomain} 
The BYU-LANL JEdif Clock Domain Parser is a jedif-based tool to parse FPGA
designs to obtain information about the clock(s). The parser first identifies
all clocks in a design. This information is then used to optionally display
other information, such as classifying Xilinx primitives into one or more 
domains, showing clock crossings, etc.

It is analagous to the EDIF-based Clock Domain Parser.

%%%%%%%%%%%%%%%%%%%%%%%%%%%%%%%%%%%%%%%%%%%%%%%%%%%%%%%%%%%%
\subsection{Toolflow Illustration}
\includegraphics[scale=0.8]{tool_flow.png}
\newpage
\section{JEdifBuild Options}
Options can be specified on the command line or in a configuration file in any 
order. This section describes each of these options in detail, which are 
summarized below:\footnote{This list is obtainable with the \texttt{--help} 
option. Here, newlines have been inserted to separate the options into 
categories.}

\begin{verbatim}
> java byucc.edif.tools.merge.JEdifBuild --help
Options:
  [-h|--help]
  [-v|--version]

  <input_file>
  [(-o|--output) <output_file>]

  [(-d|--dir) dir1,dir2,...,dirN ]
  [(-f|--include_file) include_file1,include_file2,...,include_fileN ]

  [--no_flatten]
  [--no_delete_cells]
  [--replace_srls]
  [--remove_rlocs]

\end{verbatim}

%%%%%%%%%%%%%%%%%%%%%%%%%%%%%%%%%%%%%%%%%%%%%%
\subsection{File options: input, output, etc.}
The following options specify the top-level input EDIF file, any auxiliary EDIF
files, and the destination EDIF file.

\subsubsection{\texttt{<input\_file>}}
Filename and path to the EDIF source file containing the top-level cell to be 
converted. This is the only required parameter.

Allowed filename extensions are:
\begin{itemize}
  \item Parsable Edif: edn,edf,ndf
  \item Binary Edif (Blackboxes): ngc,ngo
  \item Blackbox Utilization: bb
\end{itemize}

Parsable EDIF files will be parsed and included in the algorithms.
Binary EDIF files are not parsable, but the program recognizes them as
blackboxes, and will not complain about not finding the entity.
Blackbox utilization files allow the user to specify the resource use
of the blackboxes to help in the utilization estimate and partial tmr
algorithms.  The file format is ``Resource:Number''.  Below is an
example:

myblackbox.bb:\\
\begin{verbatim}
  BRAM:1
  FF:400
  LUT:100
\end{verbatim}

This entity, named myblackbox uses 1 BRAM, 400 Flipflops and 100 LUTS

\subsubsection{\texttt{(-o|--output) <output\_file>}}
Filename and path to the jedif output file.

Default: \texttt{BLTmr.edf} in the current working directory.

\subsubsection{\texttt{(-d|--dir) dir1,dir2,\ldots,dir3}}
Comma-separated list of directories containing external EDIF files referenced 
by the top-level EDIF file. The current working directory is included by 
default and need not be specified. There can be multiple \texttt{-d} options.

Example: \texttt{-d aux\_files,/usr/share/edif/common -d
  moreEdifFiles/}

\subsubsection{\texttt{(-f|--file) file1,file2,\ldots,fileN}}
Similar to the previous option, but rather than specifying directories to 
search, each external EDIF file is named explicitly---including the path to the 
file. There can be multiple \texttt{-f} options. 

Example: \texttt{-f multBox.edn,src/adder.edf -f /usr/share/edif/blackBox.edf}.

\subsection{Maintenance Commands}
The following options allow some control over that happens during the 
conversion process

\subsubsection{\texttt{--no\_flatten}}
By default JEdifBuild will flatten the EDIF files. Flattening is required by the 
TMR tools, but other applications my wish to have retail the hierarchical design.

\subsubsection{\texttt{--no\_delete\_cells}}
By default JEdifBuild will remove unused cells, to reduce the size of the final
.jedif file. However, the user can request that these cells be retained for 
future use.

\subsubsection{\texttt{--replace\_srls}}
By default JEdifBuild will not remove LUT RAMs acting as SRLs in the design.  These
LUT RAMs can cause problems as they are not scrubbable.  This option tells the 
tool to replace these LUT RAMs with actual SRLs.

\subsubsection{\texttt{--remove\_rlocs}}
Setting this option will tell JEdifBuild to remove ALL RLOCS in the design.  Some
RLOCS can be problems for different versions of the hardware.

%
% Words to be ignored by the spell-checker:
%

% LocalWords:  BYU LANL BLTmr EDIF FPGA TMR OBUF IBUF BUFG IBUFG LUTs
% LocalWords:  SCC SCCs FFs UCF Xilinx java JHDL netlister IOB IBUFs
% LocalWords:  OBUFs logfile INOUT TMR'd tmr txt JEdifBuild jedif edn edf dir



\newpage
\section{JEdifNetlist}
JEdifNetlist takes a jedif file and creates a standard edf file.
\begin{verbatim}

>java byucc.edif.tools.merge.JEdifNetlist
Options:
  [-h|--help]
        Print this help message to stdout and exit.

  [-v|--version]
        Print version and copyright information to stdout and exit.

  <input_file>
        Input Filename. Required.

  [(-o|--output) <output_file>]
        Filename and path to the triplicated EDIF output file created by the
        BLTmr tool. (default: <input_file>.edf)

\end{verbatim}

\newpage
\section{JEdifSterilize}
This tool determines if and how half-latch removal should be
performed in the resulting EDIF file. Half-latches are constants found
throughout FPGA designs which increase sensitivity but, if upset, \emph{cannot}
be corrected through scrubbing, because they are not determined by the
configuration bits. The BL-TMR tool can remove most of the half-latches from the
input EDIF file, prior to performing TMR, significantly increasing reliability.

\begin{verbatim}
>java edu.byu.ece.edif.jedif.JEdifSterilize
Options:
  [-h|--help]
  [-v|--version]

  [--remove_hl]
  [--hl_constant <{0|1}>]
  [--hl_use_port <hl_port_name>]
  [--pack_registers <{i|o|b|n}>]

  [--remove_fmaps]

  <input_file>
  [(-o|--output) <output_file>]

  [--no_delete_cells]

  [--log <logfile>]
  [--debug[:<debug_log>]]
  [(-V|--verbose) <{1|2|3|4|5}>]
  [--append_log]

\end{verbatim}
%%%%%%%%%%%%%%%%%%%%%%%%%%%%%%%%%%%%%%%%%%%%%%
\subsection{Half Latch removal options}

\subsubsection{\texttt{--remove\_hl}}
Remove half-latches in the input design before performing TMR.

Note: Not \emph{all} half-latches can be removed at the EDIF 
level for all architectures. Some post-processing may be necessary.

\subsubsection{\texttt{--hl\_const} $\{0,1\}$}
Sets the polarity of the half-latch constant to be used, whether an 
internally-generated constant or a top-level port. 

Valid options are \texttt{0} and \texttt{1}. Default: \texttt{0}.

\subsubsection{\texttt{--hl\_use\_port <hl\_port\_name>}}
Specify a top-level port to use in place of half-latches when 
using half-latch removal. The top-level port will have the name specified with 
this option and the polarity (1 or 0) specified with the \texttt{--hlConst} 
option.

\subsubsection{\texttt{--pack\_registers} \{i\textbar o\textbar b\textbar n\}}
By default, the BL-TMR tool treats all ports on the input EdifCell as top-level
ports (those that will be the inputs and outputs of the FPGA). The half-latch 
tool must therefore treat any FFs that will be packed into the IOBs differently
than other FFs (at least with Virtex devices). This option allows the user to
specify which IOBs the registers should be packed into: inputs (\emph{i}),
outputs (\emph{o}), both (\emph{b}), or none (\emph{n}). The default is to pack
both input and output registers.

\subsubsection{\texttt{--remove\_fmaps}}
Remove FMAPS in the input design before performing NMR.
NOTE: Not *all* FMAPS can be removed at the EDIF level for all
architectures. Some post-processing may be necessary.

\subsection{File Options}

\subsubsection{\texttt{<input\_file>}}
Filename and path to the jedif source file to be
sterilized. This is the only required parameter.

\subsubsection{\texttt{(-o|--output) <output\_file>}}
Filename and path to the sterilized jedif file. 

Default: \texttt{<inputfile>.jedif} in the current working directory.


\subsubsection{\texttt{--no\_delete\_cells}}
By default the output jedif file will remove all unused cells, to 
reduce the size of the final .jedif file. 
However, the user can request that these cells be retained for 
future use.

\subsection{Logging options}

\subsubsection{\texttt{--log <logfile>}}
Specifies an alternate file for logging output.

\subsubsection{\texttt{--debug[:<debug\_log>]}}
Specifies a file for logging the debugging output.If no file
specified, debug output is printed to the log file.

\subsubsection{\texttt{(-V|--verbose) <$\{1|2|3|4|5$\}>}}
Sets the verbosity level:
1 prints only errors, 
2 warnings, 
3 normal, 
4 log to stdout. 
5 prints debug information. 
(default: 3)

\subsubsection{\texttt{--append\_log }}
Append to the logfile instead of replacing it.



\newpage
\section{JEdifTMRAnalysis}

The following options control the manner in which the BLTmr tool performs
triplication: whether to do full or partial TMR; what specific ports, cell
types, and cell instances to be triplicated; and how to handle strongly
connected components (SCCs).

\begin{verbatim}
>java byucc.edif.tools.tmr.jedif.JEdifTMRAnalysis
[-h|--help] 
[-v|--version] 

<input_file> 
[(-o|--output) <output_file>] 
[--iob_output <output_file>] 

[--full_tmr] 

[--tmr_inports] 
[--tmr_outports] 
[--no_tmr_p port1,port2,...,portN ] 

[--tmr_c cell_type1,cell_type2,...,cell_typeN ] 
[--tmr_clk clock_domain1,clock_domain2,...,clock_domainN ] 
[--tmr_i cell_instance1,cell_instance2,...,cell_instanceN ] 

[--no_tmr_c cell_type1,cell_type2,...,cell_typeN ] 
[--no_tmr_clk clock_domain1,clock_domain2,...,clock_domainN ] 
[--no_tmr_i cell_instance1,cell_instance2,...,cell_instanceN ] 

[--no_tmr_feedback] 
[--no_tmr_input_to_feedback] 
[--no_tmr_feedback_output] 
[--no_tmr_feed_forward] 
[--no_iob_feedback] 

[--scc_sort_type <{1|2|3}>] 
[--do_scc_decomposition] 
[--input_addition_type <{1|2|3}>] 
[--output_addition_type <{1|2|3}>] 

[--merge_factor <merge_factor>] 
[--optimization_factor <optimization_factor>] 
[--factor_type <{DUF|UEF|ASUF}>] 
[--factor_value <factor_value>] 

[--use_bad_cut_conn] 

[--technology <{virtex|virtex2|virtex4}>] 
[(-p|--part) <part>] 

[--write_config <config_file>]
[--use_config <config_file>]

[--log <logfile>]
[--debug[:<debug_log>]]
[(-V|--verbose) <{1|2|3|4|5}>]
[--append_log]

\end{verbatim}

%%%%%%%%%%%%%%%%%%%%%%%%%%%%%%%%%%%%%%%%%
\subsection{File Options}

\subsubsection{\texttt{<input\_file>}}
Filename and path to the jedif source file to be
triplicated. This is the only required parameter.

\subsubsection{\texttt{(-o|--output) <output\_file>}}
Filename and path to the tmr data file.

Default: \texttt{<inputfile>.ptmr} in the current working directory

\subsubsection{\texttt{--iob\_output <output\_file>}}
Filename and path to the iob analysis file.

Default: \texttt{<inputfile>.iob} in the current working directory

%%%%%%%%%%%%%%%%%%%%%%%%%%%%%%%%%%%%%%%%%%%%
\subsection{Partial TMR Options}

\subsubsection{\texttt{--full\_tmr}}
Fully triplicate the design, skipping all partial TMR analysis. This method is 
preferred only if the design is expected to fit in the target part with full 
triplication of every resource since some time-consuming algorithms are skipped.
Resource utilization estimates will still function, stopping triplication and 
warning the user if the full triplicated design is not expected to fit in the 
target part. In this case, the partially-triplicated design will still be 
written out to file, but the user would most likely benefit from the partial 
TMR algorithms.

Note: \texttt{--full\_tmr} will triplicate all logic within the design; however, 
top-level ports are not triplicated by default. To triplicate top-level ports,
use the \texttt{--tmr\_in\-ports} and \texttt{--tmr\_outports} options.

\subsubsection{\texttt{--tmr\_inports}}
Force triplication of all top-level input ports. The resulting EDIF file will
have three input ports for every input port in the original design, with names
such as \texttt{inputPort\_tmrInstance\_0}, \texttt{inputPort\_tmrInstance\_1},
and \texttt{inputPort\_tmrInstance\_2}.

\subsubsection{\texttt{--tmr\_outports}}
Force triplication of all top-level output ports. The resulting EDIF file will
have three output ports for every output port in the original design, with names 
such as \texttt{outputPort\_tmrInstance\_0}, 
\texttt{output\-Port\_tmr\-Instance\_1}, and 
\texttt{outputPort\_tmrInstance\_2}.

\subsubsection{\texttt{--no\_tmr\_p port1,port2,\ldots,portN}}
Prevent triplication of specific top-level port(s), specified as a 
comma-separated list. Used in conjunction with \texttt{--tmr\_inports} and 
\texttt{--tmr\_outports}. For example, the following will triplicate all input 
ports except the clock and reset ports, assuming \texttt{Clk} and \texttt{rst} 
are the (case-sensitive) names of the clock and reset input ports, respectively:

\texttt{--tmr\_inports --no\_tmr\_p Clk,rst} 

\subsubsection{\texttt{--tmr\_c cell\_type1,cell\_type2,\ldots,cell\_typeN}}
Force triplication of specific cell type(s), specified as a comma-separated list. 
All instances of the types specified will be triplicated. \texttt{--tmr\_c}
takes precedence over \texttt{--no\_tmr\_c}. There can be multiple 
\texttt{--tmr} options.

Examples: 
\begin{itemize}
\item \texttt{--tmr\_c bufg,ibufg,fdc}
\item \texttt{--tmr\_c bufg,ibufg --tmr\_c fdc}
\end{itemize}

\subsubsection{\texttt{--tmr\_clk clock\_domain1,clock\_domain2,
\ldots,clock\_domainN}}
Force tripliction of the specified clock domain(s), specified as a comma-separated
list. Each clock domain whould be specified with it's full path, not including
the top level instance name, each level being separated by '/' Note:
\texttt{--no\_tmr\_clk} takes precedence over \texttt{--tmr\_clk}. There can be
multiple \texttt{--tmr\_clk} options.



\subsubsection{\texttt{--tmr\_i cell\_instance1,cell\_instance2,
\ldots,cell\_instanceN}}
Force triplication of specific cell instance(s), specified as a comma-separated
list. 
Note: \texttt{--no\_tmr\_i} takes precedence over \texttt{--tmr\_i}. There can be
multiple \texttt{--tmr\_i} options.
 
Example: \texttt{--tmr\_i clk\_bufg,multiplier16/adder16/fullAdder0}

\subsubsection{\texttt{--no\_tmr\_c cell\_type1,cell\_type2,\ldots,cell\_typeN}}
Prevent triplication of specific cell type(s), specified as a comma-separated 
list. There can be multiple \texttt{--no\_tmr\_c} options.

Example: \texttt{--no\_tmr\_c bufg,ibufg,fdc}

\subsubsection{\texttt{--no\_tmr\_clkclock\_domain1,
clock\_domain2,\ldots,clock\_domainN}}
Prevent triplication of specified clock domain(s), specified as a comma-separated 
list. There can be multiple \texttt{--no\_tmr\_c} options.

Example: \texttt{--no\_tmr\_clk clk\_c}


\subsubsection{\texttt{--no\_tmr\_i cell\_instance1,
cell\_instance2,\ldots,cell\_instanceN}}
Prevent triplication of specific cell instance(s), specified as a 
comma-separated list. There can be multiple \texttt{--no\_tmr\_i} options.

Example: \texttt{--no\_tmr\_i clk\_bufg,multiplier16/adder16/fullAdder0}

\subsubsection{\texttt{--no\_tmr\_feedback}}
Skip triplication of the feedback section of the design. 
Is it \emph{not} recommended to skip triplication of the feedback section, as 
it is the most critical section for SEU mitigation.

\subsubsection{\texttt{--no\_tmr\_input\_to\_feedback}}
Skip triplication of the portions of the design that ``feed into'' the feedback 
sections. These portions also contribute to the ``persistence'' of the design 
and should be included in triplication, when possible.

\subsubsection{\texttt{--no\_tmr\_feedback\_output}}
Skip triplication of the portions of the design which are driven by the 
feedback sections of the design.

\subsubsection{\texttt{--no\_tmr\_feed\_forward}}
Skip triplication of the portions of the design which are not related to 
feedback sections (neither drive nor are driven by the feedback sections).

\subsubsection{\texttt{--no\_iob\_feedback}}
Use this option to exclude IOBs from the feedback analysis. This is useful when
a top-level inout port is involved in feedback but by design will never be 
written to and read at the same time. Thus there is no \emph{real} feedback.
Using this option may greatly reduce the amount of feedback found in the design
and thus reduce the number of voters inserted.

%%%%%%%%%%%%%%%%%%%%%%%%%%%%%%%%%%%%%%%%%%%%%%
\subsection{SCC Options}
The following options control how BLTmr handles strongly connected components 
(SCCs) and related logic. An SCC, by definition, is a set of circuit components 
that are mutually reachable. That is, following the flow of data, every 
component in the SCC can be reached from every other. In an SCC, each component 
is related to every other component. The feedback section is defined as the 
combination of all the strongly-connected components (SCCs). The following 
options determine the order in which SCCs and related logic are triplicated as 
well as whether or not SCCs can be partitioned into smaller components.

\subsubsection{\texttt{--ssc\_sort\_type} $\{1,2,3\}$}
Choose the method the BLTmr tool uses to partially triplicate logic in the 
``feedback'' section of the design.  Option 1 triplicates the largest SCCs 
first. Option 2 triplicates the smallest first. Option 3 triplicates the SCCs 
in topological order.

This option only affects the resulting circuit if only some of the feedback 
section is triplicated. If all or none of the ``feedback'' section is 
triplicated, the three options produce identical results. The difference lies 
in what \emph{order} the logic in this section is added and thus what part of 
it is triplicated if there are not enough resources available to triplicate the 
entire section.

Valid options are \texttt{1}, \texttt{2}, and \texttt{3}. Default: \texttt{3}
(topological order).

\subsubsection{\texttt{--do\_scc\_decomposition}}
Allow portions of strongly-connected components (SCCs) to be included for 
triplication. 

By default, if a single SCC is so large that it cannot be triplicated for the 
target part, it is skipped. This option allows large SCCs to be broken up into 
smaller pieces, some of which may fit in the part. This is only useful if there 
are not enough resources to triplicate the entire set of SCCs.

\subsubsection{\texttt{--input\_addition\_type} $\{1,2,3\}$}
Select between three different algorithms to partially triplicate logic in the 
``input to feedback'' section of the design. Option 1 uses a depth-first search 
starting from the inputs to the feedback section. Option 3 uses a breadth-first 
search. Option 2 uses a combination of the two.

This option only affects the resulting circuit if only some of the input
to feedback section is triplicated. If all or none of the input to feedback 
section is triplicated, the three options produce identical results. The 
difference is in what \emph{order} the logic in this section is added and thus 
what part of it is triplicated if there are not enough resources available to 
triplicate the entire section.

Results may differ between the three addition types depending on the input 
design. It is yet not clear if one method is superior to the others in general. 

Valid options are \texttt{1}, \texttt{2}, and \texttt{3}. Default: \texttt{3} 
(breadth-first search).

\subsubsection{\texttt{--output\_addition\_type} $\{1,2,3\}$}
Similar to \texttt{--inputAdditionType}, this option applies to the logic 
in the ``feedback output'' section, that is, logic that is driven by the
feedback section.

This option only affects the resulting circuit if only some of the feedback 
output section is triplicated. It has no effect if all or none of the feedback 
output section is triplicated. As with \texttt{--inputAdditionType}, it is yet
not clear if one method is superior to the others in general.

Valid options are \texttt{1}, \texttt{2}, and \texttt{3}. Default: \texttt{3} 
(breadth-first search).

%%%%%%%%%%%%%%%%%%%%%%%%%%%%%%%%%%%%%%%%%%%%%%%%%%%%%%%%%%
\subsection{Merge Factor and Optimization Factor}
The following factors are used by the utilization tracker, which estimates the 
anticipated usage of the target chip after performing (partial) TMR\@. All factors
in this section have the precision of a Java \texttt{double}. 

\subsubsection{\texttt{--merge\_factor} $\{ 0 \leq n \leq 1 \}$ }
Used to fine-tune the estimation of logic resources in the target chip. Each 
technology has an internal, default ``merge factor'' which estimates the 
percentage of LUTs and flip-flops that will share the same slice. As this 
factor is both technology and design dependent, this option allows the user to 
specify his/her own merge factor. 

The total number of logic blocks (without taking into account optimization) is 
given by the following equation:
\begin{equation*}
\mathrm{total~logic~blocks} = FFs + LUTs - (mergeFactor * FFs)
\end{equation*}

If you need to calculate a custom mergeFactor for a specific design, use the 
following equation:
\begin{equation*}
mergeFactor = \frac{(FFs + LUTs - 2 * slices)}{FFs}
\end{equation*}

Must be between 0 and 1, inclusive. Default: 0.5.

\subsubsection{\texttt{--optimization\_factor} $\{ 0 \leq n \leq 1 \}$}
The ``optimization factor'' is used to scale down the estimate of LUTs and 
flip-flops used to account for logic optimization performed during mapping. For 
example, an optimization factor of 0.90 would assume that logic optimization 
techniques would reduce the required number of LUTs and FFs by 10\%.

We define the optimization factor to be the number of logic blocks after 
optimization divided by the number of logic blocks before optimization.  So the 
final equation for the total number of logic blocks is as follows:
\begin{equation*}
\mathrm{Estimate} = optimization\_factor * (FFs + LUTs -  mergeFactor * FFs)
\end{equation*}

Must be between 0 and 1, inclusive. Default: 0.95.

\subsubsection{\texttt{--factor\_type} $\{ \mathtt{ASUF},\mathtt{UEF},\mathtt{DUF} \}$ }
Specify the Utilization Factor Type to be used. Valid Factor Types are:

\begin{itemize}
\item ASUF 

Available Space Utilization Factor: The maximum utilization of the target part,
expressed as a percentage of the unused space on the part after the original
(non-TMR'd) design has been considered.

\item UEF 

Utilization Expansion Factor: The maximum increase in utilization of the target
part, expressed as a percentage of the utilization of the original (non-TMR'd)
design.

\item DUF 

Desired Utilization Factor: The maximum percentage of the target chip to be
utilized after performing Partial TMR.
\end{itemize}

Not case sensitive.

\subsubsection{\texttt{--factor\_value}}
Specify a single Factor Value.  The number has the precision of a Java 
\texttt{double} and is interpretted based on the Factor Type as explained above.

For example, if a design occupies 30\% of the target part prior to TMR, a DUF
of 0.50 would use 50\% of the part. An UEF of 0.50 would increase the usage by
50\%, resulting in 45\% usage of the part. An ASUF of 0.50 would use 50\% of the
available space prior to TMR, resulting in 65\% usage.

Must be greater than or equal to 0. Default: 1.0.

\subsection{\texttt{--use\_bad\_cut\_conn}}
Use bad cutgroup connectivity graph

\subsection{Target Technology and Part Options}

\subsubsection{\texttt{--technology <techname>}}
Target architecture for the triplicated design. Used to take into account 
various technology-specific properties. This argument is \emph{not} 
case-sensitive.

Valid technologies: \texttt{Virtex} and \texttt{Virtex2}. Default: 
\texttt{Virtex}.

\subsubsection{\texttt{--part <partname>}}
Target architecture for the triplicated design. Used to take into account 
part-specific properties, including the number of resources available 
in each part. Valid parts include all parts from the \emph{Virtex} and 
\emph{Virtex2} product lines, represented as a concatenation of the part name 
and package type. For example, the ``Xilinx Virtex 1000 FG680'' is represented 
as \texttt{XCV1000FG680}. This argument is \emph{not} case-sensitive.

Default: \texttt{xcv1000fg680}.
% TODO: Add list of all supported part numbers. All parts listed in
% XilinxVirtexDeviceUtilizationTracker.java and
% XilinxIIVirtexDeviceUtilizationTracker.java

%%%%%%%%%%%%%%%%%%%%%%%%%%%%%%%%%%%%%%%%%%%%%%
\subsection{Configuration File Options}
\label{config options}
The BLTmr tools can use configuration files in place of command-line parameters. 
If a parameter is specified in a configuration file, it will be passed to the 
BLTmr tool, unless it is overridden by the same argument on the command-line. 

\subsubsection{\texttt{--useConfig <config\_file>}}
\label{useConfig}
Specify a configuration file from which to read parameters.

\subsubsection{\texttt{--writeConfig[:<config\_file>]}}
Write the current set of command-line parameters to a configuration file and 
exit. The parameters will be parsed to ensure they are valid, but the BLTmr tool
will not run.  Note that only the parameters on the command-line are stored in
the configuration file. When using \texttt{--writeConfig}, any use of
\texttt{--useConfig} is ignored. This is to prevent complicated cascades
of configuration files combined with command-line options.

Examples:
\begin{itemize}
  \item \texttt{--writeConfig:JonSmith.conf} will write the command-line 
  parameters to the file \texttt{JonSmith.conf} in the current directory.
  \item \texttt{--writeConfig:/usr/lib/BLTmr/common.conf} will write the
  command-line parameters to the file \texttt{/usr/share/BLTmr/common.conf}. 
  \item See section \ref{using config}, ``Using Configuration Files,'' for
  more information.
\end{itemize}

\subsection{Logging options}

\subsubsection{\texttt{--log <logfile>}}
Specifies an alternate file for logging output.

\subsubsection{\texttt{--debug[:<debug\_log>]}}
Specifies a file for logging the debugging output.If no file
specified, debug output is printed to the log file.

\subsubsection{\texttt{(-V|--verbose) <$\{1|2|3|4|5$\}>}}
Sets the verbosity level:
1 prints only errors, 
2 warnings, 
3 normal, 
4 log to stdout. 
5 prints debug information. 
(default: 3)

\subsubsection{\texttt{--append\_log }}
Append to the logfile instead of replacing it.



\newpage
\section{JEdifCutset}
\begin{verbatim}

>java edu.byu.ece.edif.jedif.JEdifCutset 
[-h|--help] 
[-v|--version] 
<input_file> 
[(-o|--output) <output_file>] 
[--ptmr <ptmr_input_file>] 
[--iob_input <input_file>] 

[--highest_fanout_cutset] 
[--highest_ff_fanout_cutset] 

[(-p|--part) <part>] 


[--log <logfile>] 
[--debug[:<debug_log>]] 
[(-V|--verbose) <{1|2|3|4|5}>] 
[--append_log]

\end{verbatim}
%%%%%%%%%%%%%%%%%%%%%%%%%%%%%%%%%%%%%%%%%%%%%%%
\subsection{File Options}

\subsubsection{\texttt{<input\_file>}}
Filename and path to the jedif source file to be
sterilized. This is the only required parameter.

\subsubsection{\texttt{--ptmr <ptmr\_input\_file>}}
Filename and path to the tmr data file created by TMRAnalysis

Default: \texttt{<inputfile>.ptmr} in the current working directory.

\subsubsection{\texttt{(-o|--output) <output\_file>}}
Filename and path to the tmr data file appended by this tool

Default: \texttt{<inputfile>.ptmr} in the current working directory.

\subsubsection{\texttt{--iob\_input <input\_file>}}
Filename and path to the iob analysis file created by TMRAnyalysis.

%%%%%%%%%%%%%%%%%%%%%%%%%%%%%%%%%%%%%%%%%%%%%%%%%%
\subsection{Cutset Algorithms}
This tools can use several different algoriths to determine where to place
voters so that the voters block all the feed back in the design. 

\subsubsection{Default}
The default to to use the algorithm from previous releases

\subsubsection{\texttt{--highest\_fanout\_cutset}}
This algorithm finds the instance with the highest fanout in each SCC.
It then places a voter on this output. This algorithm has proven worse 
at reducing the number of voters between flipflops.

\subsubsection{\texttt{--highest\_ff\_fanout\_cutset}}
This algorighm finds the flipflop with the highest fanout in each SCC, and 
places a voter on the output. This algorithm has proven very good and reducing 
the number of paths that have more than one voter between flipflops.

%%%%%%%%%%%%%%%%%%%%%%%%%%%%%%%%%%%%%%%%%%%%%%%%%%

\subsection{Target Technology and Part Options}

\subsubsection{\texttt{--technology <techname>}}
Target architecture for the triplicated design. Used to take into account 
various technology-specific properties. This argument is \emph{not} 
case-sensitive.

Valid technologies: \texttt{Virtex} and \texttt{Virtex2}. Default: 
\texttt{Virtex}.

\subsubsection{\texttt{--part <partname>}}
Target architecture for the triplicated design. Used to take into account 
part-specific properties, including the number of resources available 
in each part. Valid parts include all parts from the \emph{Virtex} and 
\emph{Virtex2} product lines, represented as a concatenation of the part name 
and package type. For example, the ``Xilinx Virtex 1000 FG680'' is represented 
as \texttt{XCV1000FG680}. This argument is \emph{not} case-sensitive.

Default: \texttt{xcv1000fg680}.
% TODO: Add list of all supported part numbers. All parts listed in
% XilinxVirtexDeviceUtilizationTracker.java and
% XilinxIIVirtexDeviceUtilizationTracker.java

\subsection{Logging options}

\subsubsection{\texttt{--log <logfile>}}
Specifies an alternate file for logging output.

\subsubsection{\texttt{--debug[:<debug\_log>]}}
Specifies a file for logging the debugging output.If no file
specified, debug output is printed to the log file.

\subsubsection{\texttt{(-V|--verbose) <$\{1|2|3|4|5$\}>}}
Sets the verbosity level:
1 prints only errors, 
2 warnings, 
3 normal, 
4 log to stdout. 
5 prints debug information. 
(default: 3)

\subsubsection{\texttt{--append\_log }}
Append to the logfile instead of replacing it.


\newpage
\section{JEdifMoreFrequentVoting}
JEdifMoreFrequentVoting inserts extra voters for more frequent voting within
a design based on a logic levels threshold or a total number of desired
partitions.

At times, the user may wish to disable voter insertion on certain nets. This
can be accomplished by inserting a \texttt{`do\_not\_restore'} property on
selected nets in the .edf file as follows:\\
\texttt{(property do\_not\_restore (boolean (true)))}\\

Nets with this property will not have voters placed on them.

\begin{verbatim}
>java edu.byu.ece.edif.jedif.JEdifMoreFrequentVoting
Options:
  [-h|--help]
  [-v|--version]

  <input_file>
  (-r|--rep_desc) <rep_desc>

  [--voter_threshold <voter_threshold>]
  [--num_partitions <num_partitions>]
  [--voter_location_i voter_location_i1,voter_location_i2,...,voter_location_iN ]
  [--voter_location_n voter_location_n1,voter_location_n2,...,voter_location_nN ]

  [(-p|--part) <part>]

  [--write_config <config_file>]
  [--use_config <config_file>]

  [--log <logfile>]
  [--debug[:<debug_log>]]
  [(-V|--verbose) <{1|2|3|4|5}>]
  [--append_log]

\end{verbatim}

%%%%%%%%%%%%%%%%%%%%%%%%%%%%%%%%%%%%%%%%%%%%%%%%%%%%
\subsection{File Options}

\subsubsection{\texttt{<input\_file>}}
Filename and path to the jedif source file.

\subsubsection{\texttt{(-r|--rep\_desc) <rep\_desc>}}
Filename and path to the replication description (.rdesc) file to be modified.

%%%%%%%%%%%%%%%%%%%%%%%%%%%%%%%%%%%%%%%%%%%%%%%%%%%%%%%%%

\subsection{Voter Options}

\subsubsection{\texttt{--voter\_threshold <voter\_threshold>}}
Voter insertion threshold. The number of instances allowed between voters.

\subsubsection{\texttt{--num\_partitions <num\_partitions>}}
The number of partitions desired in the output circuit. The partitions are
created in equal sizes, if possible. This number is currently calculated from
the maximum depth of the circuit. A more sophisticated algorithm may be
provided in the future.

\subsubsection{\texttt{--voter\_location\_i
voter\_location\_i1,voter\_location\_i2,...,voter\_location\_iN}}
Comma-separated list of instances to place voters after. Voters will be placed
on all of the nets that are outputs of the instances in the list.

\subsubsection{\texttt{--voter\_location\_n
voter\_location\_n1,voter\_location\_n2,...,voter\_location\_nN}}
Comma-separated list of nets on which to place voters.

\subsection{Target Technology and Part Options}

\subsubsection{\texttt{--technology <techname>}}
Target architecture for the triplicated design. Used to take into account 
various technology-specific properties. This argument is \emph{not} 
case-sensitive.

Valid technologies: \texttt{Virtex} and \texttt{Virtex2}. Default: 
\texttt{Virtex}.

\subsubsection{\texttt{--part <partname>}}
Target architecture for the triplicated design. Used to take into account 
part-specific properties, including the number of resources available 
in each part. Valid parts include all parts from the \emph{Virtex} and 
\emph{Virtex2} product lines, represented as a concatenation of the part name 
and package type. For example, the ``Xilinx Virtex 1000 FG680'' is represented 
as \texttt{XCV1000FG680}. This argument is \emph{not} case-sensitive.

Default: \texttt{xcv1000fg680}.
% TODO: Add list of all supported part numbers. All parts listed in
% XilinxVirtexDeviceUtilizationTracker.java and
% XilinxIIVirtexDeviceUtilizationTracker.java

%%%%%%%%%%%%%%%%%%%%%%%%%%%%%%%%%%%%%%%%%%%%%%
\subsection{Configuration File Options}
\label{config options}
The BLTmr tools can use configuration files in place of command-line parameters. 
If a parameter is specified in a configuration file, it will be passed to the 
BLTmr tool, unless it is overridden by the same argument on the command-line. 

\subsubsection{\texttt{--useConfig <config\_file>}}
\label{useConfig}
Specify a configuration file from which to read parameters.

\subsubsection{\texttt{--writeConfig[:<config\_file>]}}
Write the current set of command-line parameters to a configuration file and 
exit. The parameters will be parsed to ensure they are valid, but the BLTmr tool
will not run.  Note that only the parameters on the command-line are stored in
the configuration file. When using \texttt{--writeConfig}, any use of
\texttt{--useConfig} is ignored. This is to prevent complicated cascades
of configuration files combined with command-line options.

Examples:
\begin{itemize}
  \item \texttt{--writeConfig:JonSmith.conf} will write the command-line 
  parameters to the file \texttt{JonSmith.conf} in the current directory.
  \item \texttt{--writeConfig:/usr/lib/BLTmr/common.conf} will write the
  command-line parameters to the file \texttt{/usr/share/BLTmr/common.conf}. 
  \item See section \ref{using config}, ``Using Configuration Files,'' for
  more information.
\end{itemize}

\subsection{Logging options}

\subsubsection{\texttt{--log <logfile>}}
Specifies an alternate file for logging output.

\subsubsection{\texttt{--debug[:<debug\_log>]}}
Specifies a file for logging the debugging output.If no file
specified, debug output is printed to the log file.

\subsubsection{\texttt{(-V|--verbose) <$\{1|2|3|4|5$\}>}}
Sets the verbosity level:
1 prints only errors, 
2 warnings, 
3 normal, 
4 log to stdout. 
5 prints debug information. 
(default: 3)

\subsubsection{\texttt{--append\_log }}
Append to the logfile instead of replacing it.


\newpage

\section{JEdifTMR}
JEdif TMR takes the data from the analysis steps and creates a 
triplicated cell

\begin{verbatim}
> java byucc.edif.tools.tmr.jedif.JEdifTMR -h
JEdifTMR - v0.3.2
Triplicates .jedif netlist according to previous triplication policy.
Copyright (c) 2005-2007 Brigham Young University, All rights reserved.

Options:
  [-h|--help]
  [-v|--version]

  <input_file>
  [--ptmr <ptmr_file>]

  [(-o|--output) <output_file>]
  [--edif]

  [--tmr_suffix suffix1,suffix2,...,suffixN ]
  [--tmr_cell_name <tmr_cell_name>]
  [--tmr_domain_report <tmr_domain_report>]

  [--tmr_ports_file <tmr_ports_file>]

  [--log <logfile>]
  [--debug[:<debug_log>]]
  [(-V|--verbose) <{1|2|3|4|5}>]
  [--append_log]


\end{verbatim}

%%%%%%%%%%%%%%%%%%%%%%%%%%%%%%%%%%%%%%%%%%%%%%
\subsection{File Options}

\subsubsection{\texttt{<input\_file>}}
Filename and path to the jedif source file to be
Triplicated. This is the only required parameter.

\subsubsection{\texttt{--ptmr <ptmr\_file>}}
Name of the tmr data file. 

Default: \texttt{<inputfile>.ptmr} in the current working directory.

\subsubsection{\texttt{(-o|--output) <output\_file>}}
Filename and path to the output file. See the \texttt{--edif}
option

Default: \texttt{<inputfile>.<type>} in the current working directory.
(Extension depends on output type. Can be overridden by including
an extension on the command line)

\subsubsection{\texttt{--edif}}
Specifies whether to output an edf file or a jedif file.

Default: generate a jedif file.

%%%%%%%%%%%%%%%%%%%%%%%%%%%%%%%%%%%%%%%%%%%55
\subsection{TMR Options}
\subsubsection{\texttt{--tmr\_suffix}}
This option allows the user to specify the suffixes that the tool adds to each
of the replicated design elements (nets, cell instances, etc.). The user must
specify all three suffixes (in a comma-separated list) to be used for each of
the three TMR domains. In addition, each of the domains must be unique to avoid
naming conflicts. One of suffixes may be left blank to signify that no suffix
should be added to the elements of that domain.

As an example, if the user specified the following list of suffixes:

\texttt{--tmrSuffix \_0,\_1,\_2}

a net in the original design called \texttt{reset} would be triplicated as:

\texttt{reset\_0, reset\_1, and reset\_2}.

The default suffixes (if the \texttt{--tmrSuffix} option is not used) are 
\texttt{\_TMR\_0, \_TMR\_1, and \_TMR\_2}.

\subsubsection{\texttt{--tmr\_cell\_name}}
 Specifies the name of the TMR'd cell


\subsubsection{\texttt{--domain\_report <domainReport>}}
The name of the domain report file. The domain report lists the domain (0, 1, 
or 2) of each cell instance in the resulting EDIF file. This report is used as 
input to the LANL RadDRC half-latch removal tool. 

Default: \texttt{BLTmr\_domain\_report.txt}.

\subsubsection{\texttt{--tmr\_ports\_file <tmr\_ports\_file>}}
The name of the tmr\_ports file. It consists of a list of ports to
triplicate followed by three names for the new ports, in this format:

\texttt{clk:clk0,clk1,clk2}
\texttt{rst0:rst0,rst1,rst2}

Default: none


\subsection{Logging options}

\subsubsection{\texttt{--log <logfile>}}
Specifies an alternate file for logging output.

\subsubsection{\texttt{--debug[:<debug\_log>]}}
Specifies a file for logging the debugging output.If no file
specified, debug output is printed to the log file.

\subsubsection{\texttt{(-V|--verbose) <$\{1|2|3|4|5$\}>}}
Sets the verbosity level:
1 prints only errors, 
2 warnings, 
3 normal, 
4 log to stdout. 
5 prints debug information. 
(default: 3)

\subsubsection{\texttt{--append\_log }}
Append to the logfile instead of replacing it.



\newpage
\section{EdifTMR}
The EdifTMR tools combines all the tools in the JEdifTMR toolflow into
one easy step. If you already know exactly what setting you wish use
to perform TMR, this tool will be faster.

\begin{verbatim}
>java byucc.edif.tools.tmr.jedif.EdifTMR
TMR - v1.0 
Runs through the JEdif TMR toolflow in one step
Copyright (c) 2005-2007 Brigham Young University, All rights reserved.

[-h|--help]
[-v|--version]

<input_file> 
[(-o|--output) <output_file>]

[(-d|--dir) dir1,dir2,...,dirN ]
[(-f|--file) file1,file2,...,fileN ]

[--remove_hl]
[--hl_constant <{0|1}>]
[--hl_port_name <hl_port_name>]
[--pack_registers <{i|o|b|n}>]
[--remove_fmaps]

[--full_tmr]
[--tmr_inports]
[--tmr_outports]
[--no_tmr_p port1,port2,...,portN ]
[--tmr_c cell_type1,cell_type2,...,cell_typeN ]
[--tmr_clk clock_domain1,clock_domain2,...,clock_domainN ]
[--tmr_i cell_instance1,cell_instance2,...,cell_instanceN ]
[--no_tmr_c cell_type1,cell_type2,...,cell_typeN ]
[--no_tmr_clk clock_domain1,clock_domain2,...,clock_domainN ]
[--no_tmr_i cell_instance1,cell_instance2,...,cell_instanceN ]
[--no_tmr_feedback]
[--no_tmr_input_to_feedback]
[--no_tmr_feedback_output]
[--no_tmr_feed_forward]
[--no_iob_feedback]

[--scc_sort_type <{1|2|3}>]
[--do_scc_decomposition]

[--input_addition_type <{1|2|3}>]
[--output_addition_type <{1|2|3}>]
[--merge_factor <merge_factor>]
[--optimization_factor <optimization_factor>]
[--factor_type <{DUF|UEF|ASUF}>]
[--factor_value <factor_value>]

[--use_bad_cut_conn]
[--highest_fanout_cutset]
[--highest_ff_fanout_cutset]


[--tmr_suffix suffix1,suffix2,...,suffixN ]
[--tmr_cell_name <tmr_cell_name>]
[--tmr_domain_report <tmr_domain_report>]


[--technology <{virtex|virtex2|virtex4}>]
[(-p|--part) <part>]

[--log <logfile>]
[--debug[:<debug_log>]]
[(-V|--verbose) <{1|2|3|4|5}>]
[--append_log]

\end{verbatim}
\subsection{Options}

This options for this tool include almost all of the options for the
various tools in the toolflow. The output option for the intermediate
files within the tool flow are not applicable, as this tool neither
creates nor uses intermediate files.

\subsubsection{File Input/Output}

The input to this tool can be either an edif file or jedif file,
depending on the extension. If the file extension is \texttt{.jedif}
it will be treated as a jedif file. All other extensions are treated
as edif files.

The \texttt{--file} and \texttt{--dir} options are ignored when the
input is a jedif file

This tool can produce either a jedif file or an edif file. The default
is edf. To generate a jedif file, make sure the output filename ends
with \texttt{.jedif}.

\newpage
\section{JEdifQuery}
JEdifQuery is used to query the contents of a .jedif file and to
provide summary information about the EDIF design contained within.
\begin{verbatim}
>java edu.byu.ece.edif.jedif.JEdifQuery
Options:
  [-h|--help]
  [-v|--version]

  <input_file>

  [--libraries]
  [--ports cell1,cell2,...,cellN ]
  [--instance_list cell_type1,cell_type2,...,cell_typeN ]
  [--cells_in_library library1,library2,...,libraryN ]
  [--subcells cell1,cell2,...,cellN ]
  [--cell_nets cell1,cell2,...,cellN ]
  [--primitive_list cell1,cell2,...,cellN ]
  [--dangling_nets]
  [--count_persistent_FFs]
\end{verbatim}

\subsection{File Options:}
\subsubsection{\texttt{<input\_file>}}
Filename and path to the .jedif source file containing the EDIF environment to
be queried. This is the only required parameter.

\subsection{Querying Options}
By default (if only an input filename is provided), JEdifQuery will
display the name of the top-level cell of the design, the number of
nets and instances contained in the design, and the top-level ports
of the design. Further information can be obtained by providing one
or more of the following command-line options:

\subsubsection{\texttt{--libraries}}
Displays a list of all libraries contained in the design.

\subsubsection{\texttt{--ports cell1,cell2,/ldots,cellN}}
All ports of specified cells contained in the design will be displayed.

\subsubsection{\texttt{--instance\_list
cell\_type1,cell\_type2,\ldots,cell\_typeN}}
The names of all instances of specified cells contained in the design will be displayed.

\subsubsection{\texttt{--cells\_in\_library library1,library2,\ldots,libraryN}}
Names of all cells contained in the specified libraries will be displayed.

\subsubsection{\texttt{--subcells cell1,cell2,\ldots,cellN}}
Names of all instances within the specified cells will be displayed.

\subsubsection{\texttt{--cell\_nets cell1,cell2,\ldots,cellN}}
Names of all nets within the specified cells will be displayed.

\subsubsection{\texttt{--primitive\_list cell1,cell2,\ldots,cellN}}
Names of all primitives within the cell will be displayed. Note that
this option only displays the names of primitives that are immediate
subcells of the specified cells; the name of a primitive that is within
a subcell of a specified subcell will not be displayed.

\subsubsection{\texttt{--dangling\_nets}}
This option displays a list of all dangling nets contained in the
design, along with the cell and library in which those nets are
contained.

\newpage

%%%%%%%%%%%%%%%%%%%%%%%%%%%%%%%%%%%%%%%%%%%%%%%%%%%%%%%%%%%%
\section{JEdifClockDomain}
JEdifClockDomain is a .jedif based tool to analyze FPGA designs to obtain
information about the clock(s). The tool first identifies all clocks in a
design. This information is then used to optionally display other information,
such as classifying Xilinx primitives into one or more domains, showing clock
crossings, etc.

Several different options can be specified on the command line in any order.
This section describes each of these options in detail.

\begin{verbatim}
> java edu.byu.ece.edif.jedif.JEdifClockDomain
Options:
  [-h|--help]
  [-v|--version]

  <input_file>
  [(-o|--output) <output_file>]

  [--domain domain1,domain2,...,domainN]
  [--show_no_domain]
  [--show_clock_crossings show_clock_crossings1,show_clock_crossings2,...,show_clock_crossingsN ]
  [--create_dotty_graph create_dotty_graph1,create_dotty_graph2,...,create_dotty_graphN ]
  [--show_cells]
  [--show_nets]
  [--show_synchronous]
  [--show_asynchronous]
  [--show_gated_clocks]
  [--show_asynchronous_resets]
  [--show_asynchronous_reset_cells]
  [--do_scc_analysis]
  [--no_iob_feedback]
\end{verbatim}

%%%%%%%%%%%%%%%%%%%%%%%%%%%%%%%%%%%%%%%%%%%%%%
\subsection{File options: input, output, etc.}

\subsubsection{\texttt{<input\_file>}}
Filename and path to the EDIF source file containing the top-level cell to be
analyzed. This is the only required parameter.

\subsubsection{\texttt{(-o) <output\_file>}}
Filename and path for the output of the tool.

Default: stdout

%%%%%%%%%%%%%%%%%%%%%%%%%%%%%%%%%%%%%%%%%%%%%%
\subsection{Display Options}
By default, all clocks in the design are displayed and nothing else. The
following options determine which additional information to display (if any).
For example, if the option  \texttt{--do\_scc\_analysis} is not specified, the
SCC algorithms will not run, decreasing runtime.

\subsubsection{\texttt{--domain}}
This option allows the user to specify which domain(s) to display.  This option
alone doesn't display any information, but acts as a filter for other options. 
The options affected by the value of this option are:
\texttt{--show\_cells, --show\_nets, --show\_synchronous, --show\_asynchronous}.
If any of these options are specified, only information for items in the
specified domain(s) is displayed. The clock domain names are case-insensitive.

For example, if the design conatins $2$ clocks (clkA and clkB) and the user
wishes to display all cells in domain clkA, the following should be run:

java edu.byu.ece.edif.jedif.JEdifClockDomain myDesign.edf \texttt{--domain clka
--show\_cells}

\subsubsection{\texttt{--show\_no\_domain}}
This option is used to include those items which don't belong to a domain. As
with the previous option, this option applies only to the following options:
\texttt{--show\_cells, --show\_nets, --show\_synchronous, --show\_asynchronous}.
An example of a cell not in a domain is an IBUF, since it is not driven by
sequential logic.

\subsubsection{\texttt{--show\_clock\_crossings clk$1$,clk$2$}}
Display all clk$1$ to clk$2$ clock crossings in the design.  A clock crossing is
defined as a net driving an input port of a sequential cell (e.g. flip-flop), 
where the net's domain differs from the sequential cell's domain.  The string
`all' can be used as a wildcard for all domains.  For example,
\texttt{--show\_clock\_crossings all,clk$1$} will display all clock crossing
from any domain to clk$1$.  Likewise, \texttt{--show\_clock\_crossings
clk$1$,all} will display all clock crossings from clk$1$ to any other domain. 
Also, \texttt{--show\_clock\_crossings all,all} can be used to show all clock
crossings of all domains.  More than one set of clock domains can be specified,
but the number of domains given must be a multiple of two.  For example,
\texttt{--show\_clock\_crossings clk$1$,clk$2$,clk$2$,clk$1$} would show all
crossings from clk$1$ to clk$2$ and from clk$2$ to clk$1$.

Example output from using this option is below:
\begin{verbatim}
clk1 to clk2 (1 crossings)
  RAM(RAMB4_S1_S1) [rsta] Net:rstZ0 Driver:[rst (FD)]
\end{verbatim}
The interpretation of this output is that the `rsta' port of a cell named RAM of
type RAMB$4$\_S$1$\_S$1$ is driven by a cell named `rst' of type FD.  The net
connecting the cells is rstZ$0$. Since this is a clk$1$ to clk$2$ crossing, the
RAM is in the clk$2$ domain and the FD is in the clk$1$ domain.

\subsubsection{\texttt{--create\_dotty\_graph cell,port}}
Generate a dotty graph representation of ``cell'' and it's predecessors.  This
may be of particular interest when a clock crossing occurs that is not flip-flop
to flip-flop. This option creates a graph starting at the cell given as a
parameter.  The graph contains predecessors only through the specified port. 
For example, if port ``d'' is chosen for a flip-flop, the clock resource will
not be shown in the graph.  The graph extends back until another synchronous
element is found, or until no predecessors remain.

\subsubsection{\texttt{--show\_cells}}
Show all primitives in the design grouped by domain. Only cells in the domains
specified by --domain are shown. Flip-flops only fall into the domain of the net
driving their clock ports.  Non-sequential cells such as LUTs may belong to
multiple domains.

\subsubsection{\texttt{--show\_nets}}
Show all nets in the design grouped by domain. Only nets in the domains 
specified by --domain are shown. Nets may fall into multiple domains.

\subsubsection{\texttt{--show\_synchronous}}
Show all synchronous primitives in the design grouped by domain. Only cells in
the domains specified by --domain are shown. Synchronous cells include
flip-flops and BRAMs.  Dual-ported BRAMs may fall into multiple domains.

\subsubsection{\texttt{--show\_asynchronous}}
Show all asynchronous primitives in the design grouped by domain. Only cells in
the domains specified by --domain are shown. Asynchronous cells are all cells
that do not have a clock input.

\subsubsection{\texttt{--show\_gated\_clocks}}
Identify all clocks (if any) that are not driven by a BUFG, DCM or DLL.

\subsubsection{\texttt{--show\_asynchronous\_resets}}
Show all flip-flops in the design that have asynchronus reset/set ports, along
with the nets driving those ports.

\subsubsection{\texttt{--show\_asynchronous\_reset\_cells}}
Show listing of all nets driving asynchronous reset ports, along with
the cells they drive.

\subsubsection{\texttt{--do\_scc\_analysis}}
Show report of SCCs in design.  This option will show the number of SCCs in the
design, as well as the size of each one and the domains contained in the SCC.

\subsubsection{\texttt{--no\_iob\_feedback}}
Use this option to exclude IOBs from the feedback analysis. This is useful when
a top-level inout port is involved in feedback but by design will never be 
written to and read at the same time. Thus there is no \emph{real} feedback.
Using this option may greatly reduce the amount of feedback found in the design.


%
% Words to be ignored by the spell-checker:
%

% LocalWords:  BYU LANL BL-TMR EDIF FPGA TMR OBUF IBUF BUFG IBUFG LUTs
% LocalWords:  SCC SCCs FFs UCF Xilinx java JHDL netlister IOB IBUFs
% LocalWords:  OBUFs logfile INOUT TMR'd tmr txt 

\newpage

\section{Common Usage of JEdifTMRAnyalysis}
This section describes a few sample scenarios and explains which combination of
command line options should be used for each.

%%%%%%%%%%%%%%%%%%%%%%%%%%%%%%%%%%%%%%%%%%%%%%
\subsection{Full TMR}
\label{subsec:fulltmr}
This example shows how to perform ``full'' TMR (triplication of all components) 
on a design. For larger designs, this may result in a TMR'd version of the 
design that does not fit in the desired chip. If this is the case, some form of 
``partial'' TMR should be used.

In this example, the design to be triplicated is specified in the file 
\texttt{myDesign.edf} and the triplicated design will be written to 
\texttt{myDesign\_tmr.edf}. Both input ports and output ports are triplicated. 
The part used in this case is the Virtex II XC2V1000-FG456.

\begin{verbatim}
> java edu.byu.ece.edif.jedif.JEdifTMRAnalysis myDesign.jedif \
  -o myDesign.ptmr -iob_output myDesign.iob\
  --tmr_inports --tmr_outports --full_tmr --technology Virtex2 --part xc2v1000fg456
\end{verbatim}


%%%%%%%%%%%%%%%%%%%%%%%%%%%%%%%%%%%%%%%%%%%%%%
\subsection{Full TMR---Clock not triplicated}
Some systems do not support triplicated clock lines. This example shows how to 
triplicate everything but the clock line.

This example is almost identical to the example in section 
\ref{subsec:fulltmr}. The \texttt{--no\_tmr\_p} option specifies that the 
top-level port named \texttt{Clk} (case-sensitive) should not be triplicated. 
The \texttt{--no\_tmr\_c} option indicates that all cells of the global clock 
buffer type \texttt{BUFG} should also not be triplicated. This prevents the 
clock line after the buffer from being triplicated and the entire circuit will 
use the same single clock.

\begin{verbatim}
> java edu.byu.ece.edif.jedif.JEdifTMRAnalysis myDesign.jedif \
  --tmr_inports --tmr_outports --full_tmr --no_tmr_p Clk --no_tmr_c BUFG \
  --technology  Virtex2 --part xc2v1000fg456
\end{verbatim}


%%%%%%%%%%%%%%%%%%%%%%%%%%%%%%%%%%%%%%%%%%%%%%
\subsection{Full TMR---No I/O triplication}
Many FPGA applications are port-limited. This example shows how to prevent all 
inputs and outputs from being triplicated. In this example, the user must leave 
out the \texttt{--tmr\_inports} and \texttt{--tmr\_outports} parameters so that 
top level ports are not included in triplication. This example also leaves out 
the \texttt{--technology} and \texttt{--part} options, using the default values 
for each (Virtex XCV1000-FG680).

\begin{verbatim}
> java edu.byu.ece.edif.jedif.JEdifTMRAnalysis myDesign.jedif --full_tmr
\end{verbatim}


%%%%%%%%%%%%%%%%%%%%%%%%%%%%%%%%%%%%%%%%%%%%%%
\subsection{Partial TMR---No I/O triplication}
This example shows a standard usage of the BL-TMR tool for partial TMR\@. In this 
case, the design is too large to fit in the targeted device when fully 
triplicated. The BL-TMR tool will triplicate as much logic as possible and 
estimate when the target chip will be fully utilized.

%This design also requires some external EDIF files that are referenced in 
%\texttt{myLargeDesign.edf} as ``black boxes.'' These external files, which are 
%located in the \texttt{externalSrcDir} directory, will be incorporated into the 
%triplicated design. Thus, the external files will not be needed with the 
%\texttt{myLargeDesign\_BL-TMR.edf} output file.

\begin{verbatim}
> java edu.byu.ece.edif.jedif.JEdifTMRAnalysis myLargeDesign.jedif 
\end{verbatim}


%%%%%%%%%%%%%%%%%%%%%%%%%%%%%%%%%%%%%%%%%%%%%%
\subsection{Partial TMR---SCC Decomposition, custom estimation factors}
In this case, the user wishes to include parts of strongly-connected components 
(SCCs) for triplication. This example also shows how to override the default
merge and optimization factors.

\begin{verbatim}
> java edu.byu.ece.edif.jedif.JEdifTMRAnalysis myLargeDesign.jedif 
  -d ./externalSrcDir --doSCCDecomposition --mergeFactor 0.4 --optimizationFactor 0.85
\end{verbatim}


%%%%%%%%%%%%%%%%%%%%%%%%%%%%%%%%%%%%%%%%%%%%%%
\subsection{Partial TMR---Fill 50\% of target device}
In some cases, the user may wish to use the triplicated design on the same chip 
as another design. In this example the user knows that a separate design will 
require half of the target chip. To fill as much of the left-over 50\% as 
possible, the user specifies a \texttt{--factor\_type} of \texttt{DUF} and a 
\texttt{factor\_value} of \texttt{0.5}. This will stop triplication of the 
input design when half of the target chip is utilized, according to the 
estimate made with the merge optimization factors.

\begin{verbatim}
> java edu.byu.ece.edif.jedif.JEdifTMRAnalysis myLargeDesign.jedif 
  --factor_type DUF --factor_value 0.5
\end{verbatim}


%%%%%%%%%%%%%%%%%%%%%%%%%%%%%%%%%%%%%%%%%%%%%%
\subsection{Partial TMR---Push utilization past 100\%}
Due to the way mapping tools are implemented, the user may be able to fit more 
logic onto the target chip than estimated by the utilization tracker. The 
Xilinx \texttt{map} program, for example, does not map unrelated logic into the 
same slice until slice utilization reaches 99\%. This means that much more 
logic can be added after this point, though the place and route step will 
become increasingly more difficult for the tools to perform.

With this in mind, to achieve the maximum capacity on the target chip, it may 
be necessary to specify a desired utilization factor greater than 1.0 (more 
than 100\% estimated utilization). The following example uses a device 
utilization factor of 1.5, which will stop triplication when an estimated 150\% 
of the target part is utilized.

\begin{verbatim}
> java edu.byu.ece.edif.jedif.JEdifTMRAnalysis myLargeDesign.jedif \
  --factor_type DUF --factor_value 1.5
\end{verbatim}


%%%%%%%%%%%%%%%%%%%%%%%%%%%%%%%%%%%%%%%%%%%%%%
\subsection{Partial TMR---Use 75\% of available space on target device}
The available space utilization factor can be used to specify the amount of
space on the target device left after the unmitigated circuit is mapped. To
fill the chip up to 75\% of the left-over space, the user specifies a 
\texttt{--factor\_type} of \texttt{ASUF} and a \texttt{factor\_value} of 
\texttt{0.75}. If the original design size is estimated at using 40\% of the
target chip, this will stop triplication when 70\% ($40 + (100-40)*0.75$) of
the target chip is utilized.

\begin{verbatim}
> java edu.byu.ece.edif.jedif.JEdifTMRAnalysis myLargeDesign.jedif 
   --factor_type DUF --factor_value 0.5
\end{verbatim}


%%%%%%%%%%%%%%%%%%%%%%%%%%%%%%%%%%%%%%%%%%%%%%
\subsection{Using Configuration Files}
\label{using config}
Configuration files can greatly simplify the use of any of the
 BL-TMR tools.  The following examples show how to create and use
 configuration files. (See section \ref{config options},
 ``Configuration File Options'' for more information.)

\subsubsection{Create a Configuration File}
The following will write the current command-line arguments to the file
\texttt{myConfig.conf}:

\begin{verbatim}
> java edu.byu.ece.edif.jedif.JEdifTMRAnalysis myDesign.edf -o myDesign_BL-TMR.edf -f \ 
subcell.edf,/usr/share/edifFiles/subcell2.edf, -d /usr/share/edif/common/ \ 
--tmr_inports --tmr_outports --tmr_c DLL,fdc,clk_buf --tmr_i clk,fifo_output \ 
--no_tmr_p clk_port,data_in --no_tmr_c bufg --no_tmr_i mux2,and24 \ 
--notmrFeedForward --inputAdditionType 1 --outputAdditionType 2 --mergeFactor \ 
0.85 --optimizationFactor 0.90 --technology Virtex2 --part XCV1000FG680 --log myLogFile.log \
--writeConfig:myConfig.conf
\end{verbatim}

The previous command creates the following output, stored in 
\texttt{myConfig.conf}:

\begin{verbatim}
#myConfig.conf, created by edu.byu.ece.edif.util.jsap.NMRCommandParser
#Sat Jul 22 19:51:14 MDT 2006
hlUsePort=hl_port
availableSpaceUtilizationFactor=0.85
tmr_c=DLL,fdc,clk_buf
summary=true
optimizationFactor=0.9
log=myLogFile.log
notmrFeedForward=true
output=myDesign_BL-TMR.edf
hlConst=0
input=myDesign.edf
outputAdditionType=2
mergeFactor=0.85
removeHL=true
dir=/usr/share/edif/common/
writeConfig=myConfig.conf
part=XCV1000FG680
no_tmr_p=clk_port,data_in
technology=Virtex2
domainReport=myDomainReport.txt
inputAdditionType=1
no_tmr_i=mux2,and24
file=subcell.edf,/usr/share/edifFiles/subcell2.edf
no_tmr_c=bufg
tmr_i=clk,fifo_output
tmr_outports=true
tmr_inports=true
\end{verbatim}

Configuration files are defined by the \texttt{java.util.Properties} class. 
However, the format is simple enough that configuration files can easily be 
created by hand or by other programs. As seen above, the format is simply 
\texttt{key=value}. A hash mark (pound sign) (\texttt{\#}) at the beginning of 
a line marks that line as a comment. BL-TMR options are given just as they would 
be on the command-line, with the exception that command-line options with no 
arguments (e.g. \texttt{--tmr\_inports}, \texttt{--summary}, 
\texttt{--doSCCDecomposition}, etc.) are specified as either \texttt{true} or 
\texttt{false}, as seen above.

\subsubsection{Use a Configuration File}

The following example shows how to load \texttt{myConfig.conf} as a configuration
file:

\begin{verbatim}
> java edu.byu.ece.edif.jedif.JEdifTMRAnalysis --useConfig myConfig.conf
\end{verbatim}

\subsubsection{Combining Configuration Files and Command-line Arguments}
Configuration files provide default values of BL-TMR options. Any options 
specified on the command-line will take precedence. (See section 
\ref{useConfig}, ``\texttt{--useConfig}'' for detailed precedence information.)
The following example uses the same options specified by myConfig.conf, but
changes the input and output files:

\begin{verbatim}
> java edu.byu.ece.edif.jedif.JEdifTMRAnalysis myOtherDesign.edf -o myOtherDesign_BL-TMR.edf \
 --useConfig myConfig.conf
\end{verbatim}

%%%%%%%%%%%%%%%%%%%%%%%%%%%%%%%%%%%%%%%%%%%%%%%%%%%%%%%%%%%%

\newpage
\section{Sample Makefile for TMR tool flow}
\verbatimtabinput{makefile}


%%%%%%%%%%%%%%%%%%%%%%%%%%%%%%%%%%%%%%%%%%%%%%

%%%%%%%%%%%%%%%%%%%%%%%%%%%%%%%%%%%%%%%%%%%%%%%%%%%%%%%%%%%%
\section{Special Notes}

%%%%%%%%%%%%%%%%%%%%%%%%%%%%%%%%%%%%%%%%%%%%%%
\subsection{Naming Conventions}
\label{naming conventions}
The BLTmr tool alters the names of triplicated signals, cell instances, and 
ports. Be aware of this when using placement (or other) constraints. An output 
port named \texttt{myOutport} in the original EDIF file will, when triplicated, 
become \texttt{myOutport\_TMR\_0, myOutport\_TMR\_1, and myOutport\_TMR\_2}. 
Similarly, a flip-flop whose instance name is \texttt{myFF} in the original file
will become \texttt{myFF\_TMR\_0}, \texttt{myFF\_TMR\_1}, 
\texttt{and myFF\_TMR\_2}.

%%%%%%%%%%%%%%%%%%%%%%%%%%%%%%%%%%%%%%%%%%%%%%
\subsection{Allocating More Memory for the JVM}
Larger designs may require more heap memory than the Java Virtual Machine (JVM) 
is allocated by default. Use the \texttt{-Xmx}
\footnote{See 
\url{http://java.sun.com/j2se/1.5.0/docs/tooldocs/windows/java.html\#Xms} 
for more information about this and other command-line options to the JVM.} 
option with the Java executable to change the maximum amount of memory for 
the virtual machine. The following example allocates up to 256 MB of 
heap space for the JVM:

\begin{verbatim}
> java -Xmx256M byucc.edif.tools.tmr.FlattenTMR ...
\end{verbatim}

%%%%%%%%%%%%%%%%%%%%%%%%%%%%%%%%%%%%%%%%%%%%%%
\subsection{RLOC Constraints}
The BLTmr tool does not support RLOC constraints at this time. RLOC constraints 
must therefore be removed before processing the resulting EDIF file. RLOCs may 
be removed from the either the source EDIF files (including any external EDIF 
files) or from the triplicated output file. Automated RLOC removal may be added 
to the tool in a later release.

To remove RLOC constraints from an EDIF file, you can use the UNIX 
\texttt{grep} tool. The following example uses \texttt{grep} to find all lines 
in the source EDIF file \texttt{myDesign\_BLTmr.edf} that do \emph{not} contain 
the string \texttt{RLOC} and copy them to the destination EDIF file 
\texttt{myDesign\_tmr\_noRLOCs.edf.} Note that you may \emph{not} use the same 
file for the source and destination.

\begin{verbatim}
> grep -v RLOC myDesign_BLTmr.edf > myDesign_BLTmr_noRLOCs.edf
\end{verbatim}

This trick is guaranteed to work for EDIF files created by the BLTmr tool as 
well as all Xilinx CoreGen EDIF files. It is \emph{not} guaranteed to work for 
all EDIF files in general since it relies on the fact that each RLOC constraint 
with its associated parentheses do not spread across multiple lines (which is 
not required by the EDIF file format). For example, it does not work for EDIF 
files created with the JHDL netlister.

%%%%%%%%%%%%%%%%%%%%%%%%%%%%%%%%%%%%%%%%%%%%%%%%%%%%%%%%%%%%
\section{Change Log}

\subsection*{Version 0.3.4 - ?? ??? 2008}
\begin{itemize}
\item Fixed --hl\_use\_port option on JEdifSterilize (was --hl\_port\_name)
\item Rearranged and added sections of this document
\item 
\end{itemize}

\subsection*{Version 0.3.3 - 15 Jan 2008}
\begin{itemize}
\item Fixed bug involving source-to-source edges in
EdifCellInstanceConnectivity. This had caused IOB registers to not be recognized
correctly.
\item Fixed bug in which Virtex II parts were incorrectly rejected by the tool
\item Added SRL Replacement to JEdifBuild
\item BLDWC now has the option to separate detection between persistent and
non-persistent
\end{itemize}

\subsection*{Version 0.3.2 - 24 Aug 2007}
\begin{itemize}
\item Various minor bug fixes
\end{itemize}

\subsection*{Version 0.3.1 - 14 Aug 2007}
\begin{itemize}
\item Fixed issues with BRAM and DSP blocks
\item BLTMR can now extract the part number if included in the input EDIF file.
\item Fixed bug in which a triplicated clock line could be voted on.
\item Fixed bug in which DLLs were not recognized.
\end{itemize}

\subsection*{Version 0.3.0 - 16 Jul 2007}
\begin{itemize}
\item Split TMR into several tools
\item TMR tool flow now uses jedif files
\item Added clock domain analysis tools
\item Added more frequent voting tool
\end{itemize}

\subsection*{Version 0.2.4 - 16 Feb 2007}
\begin{itemize}
\item Fixed bugs in EdifHalfLatchRemover: 
\begin{itemize}
\item Added support for BlackBox modules
\item Removed addition of ``\_hl'' suffix to replaced primitives
\end{itemize}
\item Added option to allow user to specify suffixes for triplicated design
elements
\item Design name in output file now matches top-level Cell
\item Fixed bug in SCCUtilities
\item created new HalfLatchFlattenedEdifCell class, tool now flattens before HL
removal
\item HL removal now recognizes IOB registers
\item Added option to ignore feedback through IOBs
\item Port components are automatically triplicated/not triplicated with the port
\end{itemize}

\subsection*{Version 0.2.3 - 12 Oct 2006}
\begin{itemize}
\item Fixed bug when writing config files
\item Fixed bug in SCC code that could cause a ConcurrentModificationException
\item Fixed bug in which BLTmr could produce an invalid design due to too many
  connections to a MUXF6 input
\end{itemize}

\subsection*{Version 0.2.2 - 25 Sep 2006}
\begin{itemize}
  \item User may now specify hierarchical instance names for forced inclusion
  in/exclusion from TMR
  \item Added Half-Latch Removal as a command-line option
  \item Now using NMREdifCell instead of TMREdifCell, which slightly changed the
  naming policy for triplicated elements (See section \ref{naming conventions}
  ``Naming Conventions'')
  \item Added fmap removal (fixes the use of inputAdditionType=1)
  \item Added Multiple EDIF Creation
  \begin{itemize}
    \item Removed the DUF, UEF, and ASUF command-line parameters
    \item Added factor\_type and factor\_value command-line parameters
  \end{itemize}
\end{itemize}

\subsection*{Version 0.2.1 - 28 Jul 2006}
\begin{itemize}
\item Now disallows voting between MUXF5, MUXF6, MUXF7, MUXF8
\item Added unused cell trimming
\end{itemize}

\subsection*{Version 0.2.0 - 25 Jul 2006}
\begin{itemize}
\item New command-line parser\footnote{The BLTmr tool uses \emph{JSAP: the 
Java-based Simple Argument Parser} by Martian Software, Inc.\ for parsing 
command-line arguments.  JSAP and its source code can be found at 
\url{http://www.martiansoftware.com/jsap/index.html}.} (not backwards 
compatible)
\item Added half-latch removal option
\item Added configuration files support
\item Some command-line parameters have been renamed
\item Added recursive black box merging
\item Fixed bug which caused some nets to lose their original name
\end{itemize}

\subsection*{Version 0.1.9 - 3 Jul 2006}
\begin{itemize}
\item Fixed bug in which output EDIF file could be invalid and crash in the map 
stage (voters were inserted in the carry chain).
\end{itemize}

\subsection*{Version 0.1.8 - 13 Jun 2006}
\begin{itemize}
\item Added triplication status to reports
\item Added available space utilization factor
\item Added force triplicate options
\item Fixed bug in SCCUtilities that could cause a class cast exception
\end{itemize}

\subsection*{Version 0.1.7 - 23 May 2006}
\begin{itemize}
\item Added DLLs to resources tracked for Virtex parts
\item Added options for ordering of SCC additions (-SCCSortType)
\end{itemize}

\subsection*{Version 0.1.6 - 18 May 2006}
\begin{itemize}
\item Added options to select type of partial Input to Feedback and Feedback 
Output addition to TMR (-input and -outputAdditionType)
\item Added option to ignore INOUT port restriction
\item Moved to Java 5
\end{itemize}

\subsection*{Version 0.1.5 - 12 May 2006}
\begin{itemize}
\item Version 0.1.4 contained a left-over debug printout. This was removed.
\end{itemize}

\subsection*{Version 0.1.4 - 11 May 2006}
\begin{itemize}
\item Added option for naming tmr domain report
\item Added utilization\_expansion\_factor option
\end{itemize}

\subsection*{Version 0.1.3 - 3 May 2006}
\begin{itemize}
\item Added options to selectively exclude Feedback, Input to Feedback, 
Feedback Output, and Feed-forward sections from TMR.
\item Complete partial TMR now matches ``Full'' TMR\@. (Before this, the 
``Feed-forward'' section was not included.)
\end{itemize}

\subsection*{Version 0.1.2 - 17 Apr 2006}
\begin{itemize}
\item Added automatic IOB handling. The user no longer needs to specifically 
include or exclude IBUFs, OBUFs, etc.
\item Added automatic log file creation. The user may also customize the 
filename of the logfile.
\item Fixed issue in which ``rename'' directives in the original EDIF were not 
preserved. The output EDIF now contains the original ``rename''s.
\end{itemize}

\subsection*{Version 0.1.1 - 9 Mar 2006}
\begin{itemize}
\item Added version number to the output of the tool for version tracking
purposes.
\end{itemize}

\subsection*{Initial Release---Version 0.1.0 - 8 Mar 2006}
\begin{itemize}
\item Initial release outside of BYU\@. No version number is contained in
the released JAR file.
\end{itemize}

\end{document}

%
% Words to be ignored by the spell-checker:
%

% LocalWords:  BYU LANL BLTmr EDIF FPGA TMR OBUF IBUF BUFG IBUFG LUTs
% LocalWords:  SCC SCCs FFs UCF Xilinx java JHDL netlister IOB IBUFs
% LocalWords:  OBUFs logfile INOUT TMR'd tmr txt 

