\section{JEdifSterilize}
This tool determines if and how half-latch removal should be
performed in the resulting EDIF file. Half-latches are constants found
throughout FPGA designs which increase sensitivity but, if upset, \emph{cannot}
be corrected through scrubbing, because they are not determined by the
configuration bits. The BL-TMR tool can remove most of the half-latches from the
input EDIF file, prior to performing TMR, significantly increasing reliability.

\begin{verbatim}
>java edu.byu.ece.edif.jedif.JEdifSterilize
Options:
  [-h|--help]
  [-v|--version]

  [--remove_hl]
  [--hl_constant <{0|1}>]
  [--hl_use_port <hl_port_name>]
  [--pack_registers <{i|o|b|n}>]

  [--remove_fmaps]

  <input_file>
  [(-o|--output) <output_file>]

  [--no_delete_cells]

  [--log <logfile>]
  [--debug[:<debug_log>]]
  [(-V|--verbose) <{1|2|3|4|5}>]
  [--append_log]

\end{verbatim}
%%%%%%%%%%%%%%%%%%%%%%%%%%%%%%%%%%%%%%%%%%%%%%
\subsection{Half Latch removal options}

\subsubsection{\texttt{--remove\_hl}}
Remove half-latches in the input design before performing TMR.

Note: Not \emph{all} half-latches can be removed at the EDIF 
level for all architectures. Some post-processing may be necessary.

\subsubsection{\texttt{--hl\_const} $\{0,1\}$}
Sets the polarity of the half-latch constant to be used, whether an 
internally-generated constant or a top-level port. 

Valid options are \texttt{0} and \texttt{1}. Default: \texttt{0}.

\subsubsection{\texttt{--hl\_use\_port <hl\_port\_name>}}
Specify a top-level port to use in place of half-latches when 
using half-latch removal. The top-level port will have the name specified with 
this option and the polarity (1 or 0) specified with the \texttt{--hlConst} 
option.

\subsubsection{\texttt{--pack\_registers} \{i\textbar o\textbar b\textbar n\}}
By default, the BL-TMR tool treats all ports on the input EdifCell as top-level
ports (those that will be the inputs and outputs of the FPGA). The half-latch 
tool must therefore treat any FFs that will be packed into the IOBs differently
than other FFs (at least with Virtex devices). This option allows the user to
specify which IOBs the registers should be packed into: inputs (\emph{i}),
outputs (\emph{o}), both (\emph{b}), or none (\emph{n}). The default is to pack
both input and output registers.

\subsubsection{\texttt{--remove\_fmaps}}
Remove FMAPS in the input design before performing NMR.
NOTE: Not *all* FMAPS can be removed at the EDIF level for all
architectures. Some post-processing may be necessary.

\subsection{File Options}

\subsubsection{\texttt{<input\_file>}}
Filename and path to the jedif source file to be
sterilized. This is the only required parameter.

\subsubsection{\texttt{(-o|--output) <output\_file>}}
Filename and path to the sterilized jedif file. 

Default: \texttt{<inputfile>.jedif} in the current working directory.


\subsubsection{\texttt{--no\_delete\_cells}}
By default the output jedif file will remove all unused cells, to 
reduce the size of the final .jedif file. 
However, the user can request that these cells be retained for 
future use.

\subsection{Logging options}

\subsubsection{\texttt{--log <logfile>}}
Specifies a file for logging output (default: sterilize.log)

\subsubsection{\texttt{--debug[:<debug\_log>]}}
Specifies a file for logging the debuggin output.If no file
specified, debug output is printed to the log file.

\subsubsection{\texttt{(-V|--verbose) <$\{1|2|3|4|5$\}>}}
Sets the verbosity level:
1 prints only errors, 
2 warnings, 
3 normal, 
4 log to stdout. 
5 prints debug information. 
(default: 3)

\subsubsection{\texttt{--append\_log }}
Append to the logfile instead of replacing it.


