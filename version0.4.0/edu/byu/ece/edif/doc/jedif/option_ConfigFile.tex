%%%%%%%%%%%%%%%%%%%%%%%%%%%%%%%%%%%%%%%%%%%%%%
\subsection{Configuration File Options}
\label{config options}
The BL-TMR tools can use configuration files in place of command-line parameters. 
If a parameter is specified in a configuration file, it will be passed to the 
BL-TMR tool, unless it is overridden by the same argument on the command-line. 

\subsubsection{\texttt{--useConfig <config\_file>}}
\label{useConfig}
Specify a configuration file from which to read parameters.

\subsubsection{\texttt{--writeConfig[:<config\_file>]}}
Write the current set of command-line parameters to a configuration file and 
exit. The parameters will be parsed to ensure they are valid, but the BL-TMR tool
will not run.  Note that only the parameters on the command-line are stored in
the configuration file. When using \texttt{--writeConfig}, any use of
\texttt{--useConfig} is ignored. This is to prevent complicated cascades
of configuration files combined with command-line options.

Examples:
\begin{itemize}
  \item \texttt{--writeConfig:JonSmith.conf} will write the command-line 
  parameters to the file \texttt{JonSmith.conf} in the current directory.
  \item \texttt{--writeConfig:/usr/lib/BL-TMR/common.conf} will write the
  command-line parameters to the file \texttt{/usr/share/BL-TMR/common.conf}. 
  \item See section \ref{using config}, ``Using Configuration Files,'' for
  more information.
\end{itemize}
