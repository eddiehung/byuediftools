\section{JEdifTMRAnalysis}

The following options control the manner in which the BLTmr tool performs
triplication: whether to do full or partial TMR; what specific ports, cell
types, and cell instances to be triplicated; and how to handle strongly
connected components (SCCs).

\begin{verbatim}
>java byucc.edif.tools.tmr.jedif.JEdifTMRAnalysis
[-h|--help] 
[-v|--version] 

<input_file> 
[(-o|--output) <output_file>] 
[--iob_output <output_file>] 

[--full_tmr] 

[--tmr_inports] 
[--tmr_outports] 
[--no_tmr_p port1,port2,...,portN ] 

[--tmr_c cell_type1,cell_type2,...,cell_typeN ] 
[--tmr_clk clock_domain1,clock_domain2,...,clock_domainN ] 
[--tmr_i cell_instance1,cell_instance2,...,cell_instanceN ] 

[--no_tmr_c cell_type1,cell_type2,...,cell_typeN ] 
[--no_tmr_clk clock_domain1,clock_domain2,...,clock_domainN ] 
[--no_tmr_i cell_instance1,cell_instance2,...,cell_instanceN ] 

[--no_tmr_feedback] 
[--no_tmr_input_to_feedback] 
[--no_tmr_feedback_output] 
[--no_tmr_feed_forward] 
[--no_iob_feedback] 

[--scc_sort_type <{1|2|3}>] 
[--do_scc_decomposition] 
[--input_addition_type <{1|2|3}>] 
[--output_addition_type <{1|2|3}>] 

[--merge_factor <merge_factor>] 
[--optimization_factor <optimization_factor>] 
[--factor_type <{DUF|UEF|ASUF}>] 
[--factor_value <factor_value>] 

[--use_bad_cut_conn] 

[--technology <{virtex|virtex2|virtex4}>] 
[(-p|--part) <part>] 

[--write_config <config_file>]
[--use_config <config_file>]

[--log <logfile>]
[--debug[:<debug_log>]]
[(-V|--verbose) <{1|2|3|4|5}>]
[--append_log]

\end{verbatim}

%%%%%%%%%%%%%%%%%%%%%%%%%%%%%%%%%%%%%%%%%
\subsection{File Options}

\subsubsection{\texttt{<input\_file>}}
Filename and path to the jedif source file to be
triplicated. This is the only required parameter.

\subsubsection{\texttt{(-o|--output) <output\_file>}}
Filename and path to the tmr data file.

Default: \texttt{<inputfile>.ptmr} in the current working directory

\subsubsection{\texttt{--iob\_output <output\_file>}}
Filename and path to the iob analysis file.

Default: \texttt{<inputfile>.iob} in the current working directory

%%%%%%%%%%%%%%%%%%%%%%%%%%%%%%%%%%%%%%%%%%%%
\subsection{Partial TMR Options}

\subsubsection{\texttt{--full\_tmr}}
Fully triplicate the design, skipping all partial TMR analysis. This method is 
preferred only if the design is expected to fit in the target part with full 
triplication of every resource since some time-consuming algorithms are skipped.
Resource utilization estimates will still function, stopping triplication and 
warning the user if the full triplicated design is not expected to fit in the 
target part. In this case, the partially-triplicated design will still be 
written out to file, but the user would most likely benefit from the partial 
TMR algorithms.

Note: \texttt{--full\_tmr} will triplicate all logic within the design; however, 
top-level ports are not triplicated by default. To triplicate top-level ports,
use the \texttt{--tmr\_in\-ports} and \texttt{--tmr\_outports} options.

\subsubsection{\texttt{--tmr\_inports}}
Force triplication of all top-level input ports. The resulting EDIF file will
have three input ports for every input port in the original design, with names
such as \texttt{inputPort\_tmrInstance\_0}, \texttt{inputPort\_tmrInstance\_1},
and \texttt{inputPort\_tmrInstance\_2}.

\subsubsection{\texttt{--tmr\_outports}}
Force triplication of all top-level output ports. The resulting EDIF file will
have three output ports for every output port in the original design, with names 
such as \texttt{outputPort\_tmrInstance\_0}, 
\texttt{output\-Port\_tmr\-Instance\_1}, and 
\texttt{outputPort\_tmrInstance\_2}.

\subsubsection{\texttt{--no\_tmr\_p port1,port2,\ldots,portN}}
Prevent triplication of specific top-level port(s), specified as a 
comma-separated list. Used in conjunction with \texttt{--tmr\_inports} and 
\texttt{--tmr\_outports}. For example, the following will triplicate all input 
ports except the clock and reset ports, assuming \texttt{Clk} and \texttt{rst} 
are the (case-sensitive) names of the clock and reset input ports, respectively:

\texttt{--tmr\_inports --no\_tmr\_p Clk,rst} 

\subsubsection{\texttt{--tmr\_c cell\_type1,cell\_type2,\ldots,cell\_typeN}}
Force triplication of specific cell type(s), specified as a comma-separated list. 
All instances of the types specified will be triplicated. \texttt{--tmr\_c}
takes precedence over \texttt{--no\_tmr\_c}. There can be multiple 
\texttt{--tmr} options.

Examples: 
\begin{itemize}
\item \texttt{--tmr\_c bufg,ibufg,fdc}
\item \texttt{--tmr\_c bufg,ibufg --tmr\_c fdc}
\end{itemize}

\subsubsection{\texttt{--tmr\_clk clock\_domain1,clock\_domain2,
\ldots,clock\_domainN}}
Force tripliction of the specified clock domain(s), specified as a comma-separated
list. Each clock domain whould be specified with it's full path, not including
the top level instance name, each level being separated by '/' Note:
\texttt{--no\_tmr\_clk} takes precedence over \texttt{--tmr\_clk}. There can be
multiple \texttt{--tmr\_clk} options.



\subsubsection{\texttt{--tmr\_i cell\_instance1,cell\_instance2,
\ldots,cell\_instanceN}}
Force triplication of specific cell instance(s), specified as a comma-separated
list. 
Note: \texttt{--no\_tmr\_i} takes precedence over \texttt{--tmr\_i}. There can be
multiple \texttt{--tmr\_i} options.
 
Example: \texttt{--tmr\_i clk\_bufg,multiplier16/adder16/fullAdder0}

\subsubsection{\texttt{--no\_tmr\_c cell\_type1,cell\_type2,\ldots,cell\_typeN}}
Prevent triplication of specific cell type(s), specified as a comma-separated 
list. There can be multiple \texttt{--no\_tmr\_c} options.

Example: \texttt{--no\_tmr\_c bufg,ibufg,fdc}

\subsubsection{\texttt{--no\_tmr\_clkclock\_domain1,
clock\_domain2,\ldots,clock\_domainN}}
Prevent triplication of specified clock domain(s), specified as a comma-separated 
list. There can be multiple \texttt{--no\_tmr\_c} options.

Example: \texttt{--no\_tmr\_clk clk\_c}


\subsubsection{\texttt{--no\_tmr\_i cell\_instance1,
cell\_instance2,\ldots,cell\_instanceN}}
Prevent triplication of specific cell instance(s), specified as a 
comma-separated list. There can be multiple \texttt{--no\_tmr\_i} options.

Example: \texttt{--no\_tmr\_i clk\_bufg,multiplier16/adder16/fullAdder0}

\subsubsection{\texttt{--no\_tmr\_feedback}}
Skip triplication of the feedback section of the design. 
Is it \emph{not} recommended to skip triplication of the feedback section, as 
it is the most critical section for SEU mitigation.

\subsubsection{\texttt{--no\_tmr\_input\_to\_feedback}}
Skip triplication of the portions of the design that ``feed into'' the feedback 
sections. These portions also contribute to the ``persistence'' of the design 
and should be included in triplication, when possible.

\subsubsection{\texttt{--no\_tmr\_feedback\_output}}
Skip triplication of the portions of the design which are driven by the 
feedback sections of the design.

\subsubsection{\texttt{--no\_tmr\_feed\_forward}}
Skip triplication of the portions of the design which are not related to 
feedback sections (neither drive nor are driven by the feedback sections).

\subsubsection{\texttt{--no\_iob\_feedback}}
Use this option to exclude IOBs from the feedback analysis. This is useful when
a top-level inout port is involved in feedback but by design will never be 
written to and read at the same time. Thus there is no \emph{real} feedback.
Using this option may greatly reduce the amount of feedback found in the design
and thus reduce the number of voters inserted.

%%%%%%%%%%%%%%%%%%%%%%%%%%%%%%%%%%%%%%%%%%%%%%
\subsection{SCC Options}
The following options control how BLTmr handles strongly connected components 
(SCCs) and related logic. An SCC, by definition, is a set of circuit components 
that are mutually reachable. That is, following the flow of data, every 
component in the SCC can be reached from every other. In an SCC, each component 
is related to every other component. The feedback section is defined as the 
combination of all the strongly-connected components (SCCs). The following 
options determine the order in which SCCs and related logic are triplicated as 
well as whether or not SCCs can be partitioned into smaller components.

\subsubsection{\texttt{--ssc\_sort\_type} $\{1,2,3\}$}
Choose the method the BLTmr tool uses to partially triplicate logic in the 
``feedback'' section of the design.  Option 1 triplicates the largest SCCs 
first. Option 2 triplicates the smallest first. Option 3 triplicates the SCCs 
in topological order.

This option only affects the resulting circuit if only some of the feedback 
section is triplicated. If all or none of the ``feedback'' section is 
triplicated, the three options produce identical results. The difference lies 
in what \emph{order} the logic in this section is added and thus what part of 
it is triplicated if there are not enough resources available to triplicate the 
entire section.

Valid options are \texttt{1}, \texttt{2}, and \texttt{3}. Default: \texttt{3}
(topological order).

\subsubsection{\texttt{--do\_scc\_decomposition}}
Allow portions of strongly-connected components (SCCs) to be included for 
triplication. 

By default, if a single SCC is so large that it cannot be triplicated for the 
target part, it is skipped. This option allows large SCCs to be broken up into 
smaller pieces, some of which may fit in the part. This is only useful if there 
are not enough resources to triplicate the entire set of SCCs.

\subsubsection{\texttt{--input\_addition\_type} $\{1,2,3\}$}
Select between three different algorithms to partially triplicate logic in the 
``input to feedback'' section of the design. Option 1 uses a depth-first search 
starting from the inputs to the feedback section. Option 3 uses a breadth-first 
search. Option 2 uses a combination of the two.

This option only affects the resulting circuit if only some of the input
to feedback section is triplicated. If all or none of the input to feedback 
section is triplicated, the three options produce identical results. The 
difference is in what \emph{order} the logic in this section is added and thus 
what part of it is triplicated if there are not enough resources available to 
triplicate the entire section.

Results may differ between the three addition types depending on the input 
design. It is yet not clear if one method is superior to the others in general. 

Valid options are \texttt{1}, \texttt{2}, and \texttt{3}. Default: \texttt{3} 
(breadth-first search).

\subsubsection{\texttt{--output\_addition\_type} $\{1,2,3\}$}
Similar to \texttt{--inputAdditionType}, this option applies to the logic 
in the ``feedback output'' section, that is, logic that is driven by the
feedback section.

This option only affects the resulting circuit if only some of the feedback 
output section is triplicated. It has no effect if all or none of the feedback 
output section is triplicated. As with \texttt{--inputAdditionType}, it is yet
not clear if one method is superior to the others in general.

Valid options are \texttt{1}, \texttt{2}, and \texttt{3}. Default: \texttt{3} 
(breadth-first search).

%%%%%%%%%%%%%%%%%%%%%%%%%%%%%%%%%%%%%%%%%%%%%%%%%%%%%%%%%%
\subsection{Merge Factor and Optimization Factor}
The following factors are used by the utilization tracker, which estimates the 
anticipated usage of the target chip after performing (partial) TMR\@. All factors
in this section have the precision of a Java \texttt{double}. 

\subsubsection{\texttt{--merge\_factor} $\{ 0 \leq n \leq 1 \}$ }
Used to fine-tune the estimation of logic resources in the target chip. Each 
technology has an internal, default ``merge factor'' which estimates the 
percentage of LUTs and flip-flops that will share the same slice. As this 
factor is both technology and design dependent, this option allows the user to 
specify his/her own merge factor. 

The total number of logic blocks (without taking into account optimization) is 
given by the following equation:
\begin{equation*}
\mathrm{total~logic~blocks} = FFs + LUTs - (mergeFactor * FFs)
\end{equation*}

If you need to calculate a custom mergeFactor for a specific design, use the 
following equation:
\begin{equation*}
mergeFactor = \frac{(FFs + LUTs - 2 * slices)}{FFs}
\end{equation*}

Must be between 0 and 1, inclusive. Default: 0.5.

\subsubsection{\texttt{--optimization\_factor} $\{ 0 \leq n \leq 1 \}$}
The ``optimization factor'' is used to scale down the estimate of LUTs and 
flip-flops used to account for logic optimization performed during mapping. For 
example, an optimization factor of 0.90 would assume that logic optimization 
techniques would reduce the required number of LUTs and FFs by 10\%.

We define the optimization factor to be the number of logic blocks after 
optimization divided by the number of logic blocks before optimization.  So the 
final equation for the total number of logic blocks is as follows:
\begin{equation*}
\mathrm{Estimate} = optimization\_factor * (FFs + LUTs -  mergeFactor * FFs)
\end{equation*}

Must be between 0 and 1, inclusive. Default: 0.95.

\subsubsection{\texttt{--factor\_type} $\{ \mathtt{ASUF},\mathtt{UEF},\mathtt{DUF} \}$ }
Specify the Utilization Factor Type to be used. Valid Factor Types are:

\begin{itemize}
\item ASUF 

Available Space Utilization Factor: The maximum utilization of the target part,
expressed as a percentage of the unused space on the part after the original
(non-TMR'd) design has been considered.

\item UEF 

Utilization Expansion Factor: The maximum increase in utilization of the target
part, expressed as a percentage of the utilization of the original (non-TMR'd)
design.

\item DUF 

Desired Utilization Factor: The maximum percentage of the target chip to be
utilized after performing Partial TMR.
\end{itemize}

Not case sensitive.

\subsubsection{\texttt{--factor\_value}}
Specify a single Factor Value.  The number has the precision of a Java 
\texttt{double} and is interpretted based on the Factor Type as explained above.

For example, if a design occupies 30\% of the target part prior to TMR, a DUF
of 0.50 would use 50\% of the part. An UEF of 0.50 would increase the usage by
50\%, resulting in 45\% usage of the part. An ASUF of 0.50 would use 50\% of the
available space prior to TMR, resulting in 65\% usage.

Must be greater than or equal to 0. Default: 1.0.

\subsection{\texttt{--use\_bad\_cut\_conn}}
Use bad cutgroup connectivity graph

\subsection{Target Technology and Part Options}

\subsubsection{\texttt{--technology <techname>}}
Target architecture for the triplicated design. Used to take into account 
various technology-specific properties. This argument is \emph{not} 
case-sensitive.

Valid technologies: \texttt{Virtex} and \texttt{Virtex2}. Default: 
\texttt{Virtex}.

\subsubsection{\texttt{--part <partname>}}
Target architecture for the triplicated design. Used to take into account 
part-specific properties, including the number of resources available 
in each part. Valid parts include all parts from the \emph{Virtex} and 
\emph{Virtex2} product lines, represented as a concatenation of the part name 
and package type. For example, the ``Xilinx Virtex 1000 FG680'' is represented 
as \texttt{XCV1000FG680}. This argument is \emph{not} case-sensitive.

Default: \texttt{xcv1000fg680}.
% TODO: Add list of all supported part numbers. All parts listed in
% XilinxVirtexDeviceUtilizationTracker.java and
% XilinxIIVirtexDeviceUtilizationTracker.java

%%%%%%%%%%%%%%%%%%%%%%%%%%%%%%%%%%%%%%%%%%%%%%
\subsection{Configuration File Options}
\label{config options}
The BLTmr tools can use configuration files in place of command-line parameters. 
If a parameter is specified in a configuration file, it will be passed to the 
BLTmr tool, unless it is overridden by the same argument on the command-line. 

\subsubsection{\texttt{--useConfig <config\_file>}}
\label{useConfig}
Specify a configuration file from which to read parameters.

\subsubsection{\texttt{--writeConfig[:<config\_file>]}}
Write the current set of command-line parameters to a configuration file and 
exit. The parameters will be parsed to ensure they are valid, but the BLTmr tool
will not run.  Note that only the parameters on the command-line are stored in
the configuration file. When using \texttt{--writeConfig}, any use of
\texttt{--useConfig} is ignored. This is to prevent complicated cascades
of configuration files combined with command-line options.

Examples:
\begin{itemize}
  \item \texttt{--writeConfig:JonSmith.conf} will write the command-line 
  parameters to the file \texttt{JonSmith.conf} in the current directory.
  \item \texttt{--writeConfig:/usr/lib/BLTmr/common.conf} will write the
  command-line parameters to the file \texttt{/usr/share/BLTmr/common.conf}. 
  \item See section \ref{using config}, ``Using Configuration Files,'' for
  more information.
\end{itemize}

\subsection{Logging options}

\subsubsection{\texttt{--log <logfile>}}
Specifies a file for logging output (default: sterilize.log)

\subsubsection{\texttt{--debug[:<debug\_log>]}}
Specifies a file for logging the debuggin output.If no file
specified, debug output is printed to the log file.

\subsubsection{\texttt{(-V|--verbose) <$\{1|2|3|4|5$\}>}}
Sets the verbosity level:
1 prints only errors, 
2 warnings, 
3 normal, 
4 log to stdout. 
5 prints debug information. 
(default: 3)

\subsubsection{\texttt{--append\_log }}
Append to the logfile instead of replacing it.


