\section{JEdifReplicationQuery}
JEdifReplicationQuery is used to query the contents of a replication
description (.rdesc) file and to provide information about the type(s) of
replication that will be applied to a design given the information in the file.

The tool gives information about each of the replication types (i.e.
triplication, duplication) used in the design. The ports and instances selected
for each type are displayed.

The tool also gives information about organs (i.e. voters, comparators) that
will be inserted into the design on each net. An organ summary is provided that
lists the total number of each kind of organ to be inserted.

Finally, the tool lists any detection outputs to be used as well as information
about whether an output register (and which clock net) and output buffer will
be used. A list of nets that will be detected on is given for each detection
output.

\begin{verbatim}
>java edu.byu.ece.edif.jedif.JEdifReplicationQuery
Options:
  [-h|--help]
  [-v|--version]

  <input_file>
  (-r|--rep_desc) <rep_desc>
\end{verbatim}

\subsection{File Options:}
\subsubsection{\texttt{<input\_file>}}
Filename and path to the .jedif source file containing the EDIF environment to
be queried.

\subsubsection{\texttt{(-r|--rep\_desc) <rep\_desc>}}
Filename and path to the replication description (.rdesc) file containing the
replication information.
