\section{Common Usage of JEdifTMRAnyalysis}
This section describes a few sample scenarios and explains which combination of
command line options should be used for each.

%%%%%%%%%%%%%%%%%%%%%%%%%%%%%%%%%%%%%%%%%%%%%%
\subsection{Full TMR}
\label{subsec:fulltmr}
This example shows how to perform ``full'' TMR (triplication of all components) 
on a design. For larger designs, this may result in a TMR'd version of the 
design that does not fit in the desired chip. If this is the case, some form of 
``partial'' TMR should be used.

In this example, the design to be triplicated is specified in the file 
\texttt{myDesign.edf} and the triplicated design will be written to 
\texttt{myDesign\_tmr.edf}. Both input ports and output ports are triplicated. 
The part used in this case is the Virtex II XC2V1000-FG456.

\begin{verbatim}
> java edu.byu.ece.edif.jedif.JEdifTMRAnalysis myDesign.jedif \
  -o myDesign.ptmr -iob_output myDesign.iob\
  --tmr_inports --tmr_outports --full_tmr --technology Virtex2 --part xc2v1000fg456
\end{verbatim}


%%%%%%%%%%%%%%%%%%%%%%%%%%%%%%%%%%%%%%%%%%%%%%
\subsection{Full TMR---Clock not triplicated}
Some systems do not support triplicated clock lines. This example shows how to 
triplicate everything but the clock line.

This example is almost identical to the example in section 
\ref{subsec:fulltmr}. The \texttt{--no\_tmr\_p} option specifies that the 
top-level port named \texttt{Clk} (case-sensitive) should not be triplicated. 
The \texttt{--no\_tmr\_c} option indicates that all cells of the global clock 
buffer type \texttt{BUFG} should also not be triplicated. This prevents the 
clock line after the buffer from being triplicated and the entire circuit will 
use the same single clock.

\begin{verbatim}
> java edu.byu.ece.edif.jedif.JEdifTMRAnalysis myDesign.jedif \
  --tmr_inports --tmr_outports --full_tmr --no_tmr_p Clk --no_tmr_c BUFG \
  --technology  Virtex2 --part xc2v1000fg456
\end{verbatim}


%%%%%%%%%%%%%%%%%%%%%%%%%%%%%%%%%%%%%%%%%%%%%%
\subsection{Full TMR---No I/O triplication}
Many FPGA applications are port-limited. This example shows how to prevent all 
inputs and outputs from being triplicated. In this example, the user must leave 
out the \texttt{--tmr\_inports} and \texttt{--tmr\_outports} parameters so that 
top level ports are not included in triplication. This example also leaves out 
the \texttt{--technology} and \texttt{--part} options, using the default values 
for each (Virtex XCV1000-FG680).

\begin{verbatim}
> java edu.byu.ece.edif.jedif.JEdifTMRAnalysis myDesign.jedif --full_tmr
\end{verbatim}


%%%%%%%%%%%%%%%%%%%%%%%%%%%%%%%%%%%%%%%%%%%%%%
\subsection{Partial TMR---No I/O triplication}
This example shows a standard usage of the BL-TMR tool for partial TMR\@. In this 
case, the design is too large to fit in the targeted device when fully 
triplicated. The BL-TMR tool will triplicate as much logic as possible and 
estimate when the target chip will be fully utilized.

%This design also requires some external EDIF files that are referenced in 
%\texttt{myLargeDesign.edf} as ``black boxes.'' These external files, which are 
%located in the \texttt{externalSrcDir} directory, will be incorporated into the 
%triplicated design. Thus, the external files will not be needed with the 
%\texttt{myLargeDesign\_BL-TMR.edf} output file.

\begin{verbatim}
> java edu.byu.ece.edif.jedif.JEdifTMRAnalysis myLargeDesign.jedif 
\end{verbatim}


%%%%%%%%%%%%%%%%%%%%%%%%%%%%%%%%%%%%%%%%%%%%%%
\subsection{Partial TMR---SCC Decomposition, custom estimation factors}
In this case, the user wishes to include parts of strongly-connected components 
(SCCs) for triplication. This example also shows how to override the default
merge and optimization factors.

\begin{verbatim}
> java edu.byu.ece.edif.jedif.JEdifTMRAnalysis myLargeDesign.jedif 
  -d ./externalSrcDir --doSCCDecomposition --mergeFactor 0.4 --optimizationFactor 0.85
\end{verbatim}


%%%%%%%%%%%%%%%%%%%%%%%%%%%%%%%%%%%%%%%%%%%%%%
\subsection{Partial TMR---Fill 50\% of target device}
In some cases, the user may wish to use the triplicated design on the same chip 
as another design. In this example the user knows that a separate design will 
require half of the target chip. To fill as much of the left-over 50\% as 
possible, the user specifies a \texttt{--factor\_type} of \texttt{DUF} and a 
\texttt{factor\_value} of \texttt{0.5}. This will stop triplication of the 
input design when half of the target chip is utilized, according to the 
estimate made with the merge optimization factors.

\begin{verbatim}
> java edu.byu.ece.edif.jedif.JEdifTMRAnalysis myLargeDesign.jedif 
  --factor_type DUF --factor_value 0.5
\end{verbatim}


%%%%%%%%%%%%%%%%%%%%%%%%%%%%%%%%%%%%%%%%%%%%%%
\subsection{Partial TMR---Push utilization past 100\%}
Due to the way mapping tools are implemented, the user may be able to fit more 
logic onto the target chip than estimated by the utilization tracker. The 
Xilinx \texttt{map} program, for example, does not map unrelated logic into the 
same slice until slice utilization reaches 99\%. This means that much more 
logic can be added after this point, though the place and route step will 
become increasingly more difficult for the tools to perform.

With this in mind, to achieve the maximum capacity on the target chip, it may 
be necessary to specify a desired utilization factor greater than 1.0 (more 
than 100\% estimated utilization). The following example uses a device 
utilization factor of 1.5, which will stop triplication when an estimated 150\% 
of the target part is utilized.

\begin{verbatim}
> java edu.byu.ece.edif.jedif.JEdifTMRAnalysis myLargeDesign.jedif \
  --factor_type DUF --factor_value 1.5
\end{verbatim}


%%%%%%%%%%%%%%%%%%%%%%%%%%%%%%%%%%%%%%%%%%%%%%
\subsection{Partial TMR---Use 75\% of available space on target device}
The available space utilization factor can be used to specify the amount of
space on the target device left after the unmitigated circuit is mapped. To
fill the chip up to 75\% of the left-over space, the user specifies a 
\texttt{--factor\_type} of \texttt{ASUF} and a \texttt{factor\_value} of 
\texttt{0.75}. If the original design size is estimated at using 40\% of the
target chip, this will stop triplication when 70\% ($40 + (100-40)*0.75$) of
the target chip is utilized.

\begin{verbatim}
> java edu.byu.ece.edif.jedif.JEdifTMRAnalysis myLargeDesign.jedif 
   --factor_type DUF --factor_value 0.5
\end{verbatim}


%%%%%%%%%%%%%%%%%%%%%%%%%%%%%%%%%%%%%%%%%%%%%%
\subsection{Using Configuration Files}
\label{using config}
Configuration files can greatly simplify the use of any of the
 BL-TMR tools.  The following examples show how to create and use
 configuration files. (See section \ref{config options},
 ``Configuration File Options'' for more information.)

\subsubsection{Create a Configuration File}
The following will write the current command-line arguments to the file
\texttt{myConfig.conf}:

\begin{verbatim}
> java edu.byu.ece.edif.jedif.JEdifTMRAnalysis myDesign.edf -o myDesign_BL-TMR.edf -f \ 
subcell.edf,/usr/share/edifFiles/subcell2.edf, -d /usr/share/edif/common/ \ 
--tmr_inports --tmr_outports --tmr_c DLL,fdc,clk_buf --tmr_i clk,fifo_output \ 
--no_tmr_p clk_port,data_in --no_tmr_c bufg --no_tmr_i mux2,and24 \ 
--notmrFeedForward --inputAdditionType 1 --outputAdditionType 2 --mergeFactor \ 
0.85 --optimizationFactor 0.90 --technology Virtex2 --part XCV1000FG680 --log myLogFile.log \
--writeConfig:myConfig.conf
\end{verbatim}

The previous command creates the following output, stored in 
\texttt{myConfig.conf}:

\begin{verbatim}
#myConfig.conf, created by edu.byu.ece.edif.util.jsap.NMRCommandParser
#Sat Jul 22 19:51:14 MDT 2006
hlUsePort=hl_port
availableSpaceUtilizationFactor=0.85
tmr_c=DLL,fdc,clk_buf
summary=true
optimizationFactor=0.9
log=myLogFile.log
notmrFeedForward=true
output=myDesign_BL-TMR.edf
hlConst=0
input=myDesign.edf
outputAdditionType=2
mergeFactor=0.85
removeHL=true
dir=/usr/share/edif/common/
writeConfig=myConfig.conf
part=XCV1000FG680
no_tmr_p=clk_port,data_in
technology=Virtex2
domainReport=myDomainReport.txt
inputAdditionType=1
no_tmr_i=mux2,and24
file=subcell.edf,/usr/share/edifFiles/subcell2.edf
no_tmr_c=bufg
tmr_i=clk,fifo_output
tmr_outports=true
tmr_inports=true
\end{verbatim}

Configuration files are defined by the \texttt{java.util.Properties} class. 
However, the format is simple enough that configuration files can easily be 
created by hand or by other programs. As seen above, the format is simply 
\texttt{key=value}. A hash mark (pound sign) (\texttt{\#}) at the beginning of 
a line marks that line as a comment. BL-TMR options are given just as they would 
be on the command-line, with the exception that command-line options with no 
arguments (e.g. \texttt{--tmr\_inports}, \texttt{--summary}, 
\texttt{--doSCCDecomposition}, etc.) are specified as either \texttt{true} or 
\texttt{false}, as seen above.

\subsubsection{Use a Configuration File}

The following example shows how to load \texttt{myConfig.conf} as a configuration
file:

\begin{verbatim}
> java edu.byu.ece.edif.jedif.JEdifTMRAnalysis --useConfig myConfig.conf
\end{verbatim}

\subsubsection{Combining Configuration Files and Command-line Arguments}
Configuration files provide default values of BL-TMR options. Any options 
specified on the command-line will take precedence. (See section 
\ref{useConfig}, ``\texttt{--useConfig}'' for detailed precedence information.)
The following example uses the same options specified by myConfig.conf, but
changes the input and output files:

\begin{verbatim}
> java edu.byu.ece.edif.jedif.JEdifTMRAnalysis myOtherDesign.edf -o myOtherDesign_BL-TMR.edf \
 --useConfig myConfig.conf
\end{verbatim}

%%%%%%%%%%%%%%%%%%%%%%%%%%%%%%%%%%%%%%%%%%%%%%%%%%%%%%%%%%%%
