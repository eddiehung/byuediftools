\section{JEdifDetectionSelection}
JEdifDetection determines detector locations for both triplicated and
duplicated design portions using user-specified options. Like
JEdifNMRSelection, this tool is designed to be run in multiple passes
(only one replication type can be processed per pass). Results are
saved in the replication description file (.rdesc).

JEdifDetection is capable of inserting both single- and dual-rail detectors. By
default, output registers and output buffers are placed at detection signal
outputs, but this behavior can be disabled (i.e. if the design being replicated
is not a top-level design).

At times, the user may wish to force detection on certain nets and prevent
detector insertion on others. This can be accomplished by inserting
\texttt{`force\_detect'} and \texttt{`do\_not\_detect'} properties on selected
nets in the .edf file as follows:\\
\texttt{(property force\_detect (boolean (true)))}
\texttt{(property do\_not\_detect (boolean (true)))}\\

\begin{verbatim}
>java edu.byu.ece.edif.jedif.JEdifDetectionSelection
Options:
  [-h|--help]
  [-v|--version]

  <input_file>
  (-r|--rep_desc) <rep_desc>
  (-c|--c_desc) <c_desc>

  --replication_type <replication_type>
  [--rail_type <rail_type>]
  (-p|--port_name) <port_name>
  [--no_downscale_detection]
  [--no_upscale_detection]
  [--no_output_detection]
  [--no_obufs]
  [--no_oregs]
  [--clock_net <clock_net>]
 
  [--write_config <config_file>]
  [--use_config <config_file>]

  [--log <logfile>]
  [--debug[:<debug_log>]]
  [(-V|--verbose) <{1|2|3|4|5}>]
  [--append_log]
\end{verbatim}
%%%%%%%%%%%%%%%%%%%%%%%%%%%%%%%%%%%%%%%%%%%%%%%
\subsection{File Options}

\subsubsection{\texttt{<input\_file>}}
Filename and path to the jedif source file to be
sterilized. This is the only required parameter.

\subsubsection{\texttt{(-r|--rep\_desc) <rep\_desc>}}
Filename and path to the replication description (.rdesc) file to be modified.

\subsubsection{\texttt{(-c|--c\_desc) <c\_desc>}}
Filename and path to the circuit description (.cdesc) file generated by
JEdifAnalyze.

%%%%%%%%%%%%%%%%%%%%%%%%%%%%%%%%%%%%%%%%%%%%%%%%%%
\subsection{Detection Options}

\subsubsection{\texttt{--replication\_type <replication\_type>}}
Replication type to use for the current pass. Must be one of the following:
\texttt{triplication}, \texttt{duplication}.

\subsubsection{\texttt{--rail\_type <rail\_type>}}
Rail type. Must be one of the following: \texttt{single}, \texttt{dual}. The
default rail type is \texttt{single}. Single-rail detectors produce a $1$-bit
error code that is high when an error is detected. A dual-rail detector produces
a $2$-bit error code output that enables detection of comparator errors. The
`\texttt{$00$}' code indicates that no error has been detected. The
`\texttt{$11$}' code indicates that an error has been detected. The
`\texttt{$01$}' and `\texttt{$10$}' codes indicate that a comparator error has
been detected.

\subsubsection{\texttt{(-p|--port\_name) <port\_name>}}
Name of the port that should receive the detection error signals. If a port
with this name does not exist, it will be created. If the given port already
exists, it must have the correct bit-width ($1$ for single-rail detection, $2$
for dual-rail detection) or an error will occur. JEdifDetectionSelection may be
run multiple times with different port names or with the same port name. The
results of all runs with the same port name will be merged into the port.

\subsubsection{\texttt{--no\_downscale\_detection}}
This option disables the default behavior if inserting detectors at locations
where the replication factor downscales (i.e. data flows from a triplicated
partition to a duplicated partition).

\subsubsection{\texttt{--no\_upscale\_detection}}
This option disables the default behavior of inserting detectors at locations
where the replication factor upscales (i.e. data flows from a duplicated
partition to a triplicated partition).

\subsubsection{\texttt{--no\_output\_detection}}
This option disables the default behavior of inserting detectors at circuit
outputs.

\subsubsection{\texttt{--no\_obufs}}
This option disables the defualt behavior of inserting output buffers on the
detection error signal outputs. This could be useful if the tool is not
operating on a top-level design.

\subsubsection{\texttt{--no\_oregs}}
This option disables the default behavior of inserting output registers on the
detection error signal outputs.

\subsubsection{\texttt{--clock\_net <clock\_net>}}
This option specifies a clock net to use for output registers. This option is
required unless output register insertion is disabled with the
\texttt{--no\_oregs} option. The name given should be the name of the clock net
\emph{after} replication (if different).

%%%%%%%%%%%%%%%%%%%%%%%%%%%%%%%%%%%%%%%%%%%%%%%%%%

%%%%%%%%%%%%%%%%%%%%%%%%%%%%%%%%%%%%%%%%%%%%%%
\subsection{Configuration File Options}
\label{config options}
The BLTmr tools can use configuration files in place of command-line parameters. 
If a parameter is specified in a configuration file, it will be passed to the 
BLTmr tool, unless it is overridden by the same argument on the command-line. 

\subsubsection{\texttt{--useConfig <config\_file>}}
\label{useConfig}
Specify a configuration file from which to read parameters.

\subsubsection{\texttt{--writeConfig[:<config\_file>]}}
Write the current set of command-line parameters to a configuration file and 
exit. The parameters will be parsed to ensure they are valid, but the BLTmr tool
will not run.  Note that only the parameters on the command-line are stored in
the configuration file. When using \texttt{--writeConfig}, any use of
\texttt{--useConfig} is ignored. This is to prevent complicated cascades
of configuration files combined with command-line options.

Examples:
\begin{itemize}
  \item \texttt{--writeConfig:JonSmith.conf} will write the command-line 
  parameters to the file \texttt{JonSmith.conf} in the current directory.
  \item \texttt{--writeConfig:/usr/lib/BLTmr/common.conf} will write the
  command-line parameters to the file \texttt{/usr/share/BLTmr/common.conf}. 
  \item See section \ref{using config}, ``Using Configuration Files,'' for
  more information.
\end{itemize}

\subsection{Logging options}

\subsubsection{\texttt{--log <logfile>}}
Specifies a file for logging output (default: sterilize.log)

\subsubsection{\texttt{--debug[:<debug\_log>]}}
Specifies a file for logging the debuggin output.If no file
specified, debug output is printed to the log file.

\subsubsection{\texttt{(-V|--verbose) <$\{1|2|3|4|5$\}>}}
Sets the verbosity level:
1 prints only errors, 
2 warnings, 
3 normal, 
4 log to stdout. 
5 prints debug information. 
(default: 3)

\subsubsection{\texttt{--append\_log }}
Append to the logfile instead of replacing it.

